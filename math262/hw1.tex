\documentclass{article}
\usepackage{geometry}
\usepackage[namelimits,sumlimits]{amsmath}
\usepackage{amssymb,amsfonts}
\usepackage{multicol}
\usepackage{mathrsfs}
\newcommand{\nc}{\newcommand}
\newcommand{\tab}{\hspace*{5em}}
\newcommand{\conj}{\overline}
\newcommand{\dd}{\partial}
\nc{\cn}{\mathbb{C}}
\nc{\rn}{\mathbb{R}}
\nc{\pd}[2]{\frac{\partial {#1}}{\partial {#2}}}
\nc{\ep}{\epsilon}
\nc{\topo}{\mathscr{T}}
\nc{\basis}{\mathscr{B}}
\nc{\nullset}{\varnothing}
\setlength{\parindent}{0mm}
\begin{document}
Name: Hall Liu

Date: \today 
\vspace{1.5cm}

\subsection*{83.4}
a. $X$ and $\varnothing$ are members of all $\topo_\alpha$ in the family, so they're in the intersection. For an arbitrary collection of sets $A_i$ in $\bigcap\topo_\alpha$, they're each in all the $\topo_\alpha$, so their union is also in each of the $\topo_\alpha$, so the intersection is closed under arbitrary union. By similar reasoning with a finite collection of $A_i$, the intersection is closed under finite intersection and therefore a topology. $\bigcup\topo_\alpha$ is not necessarily a topology. Consider the set $A$ of $3$ elements represented as $\{1,2,3\}$. The topologies $\{\nullset,\{1\},A\}$ and $\{\nullset,\{2\},A\}$ have union $\{\nullset,\{1\},\{2\},A\}$, which is not closed under union since $\{1,2\}$ is not in it.

b. Consider a chain of topologies $X\supset\topo_1\supset\topo_2\supset\cdots$ all containing the aforementioned family. By (a), their intersection is also a topology which contains the family but is included in all other topologies in the chain. By Zorn's lemma, the collection of topologies which contain the family has minimal elements. Consider two such minimal elements which are distinct. Their intersection is yet another topology which contains the family, but is strictly included in both the minimal elements, contradicting their minimality. Thus the minimal element is unique. The unique largest topology contained in all $\topo_\alpha$ in the family is simply their intersection, since any topology larger than this must contain subsets which are not in at least one of the $\topo_\alpha$.

c. Smallest containing: $\{\nullset,\{a\},\{b\},\{a,b\},\{b,c\},X\}$. Largest contained: $\{\nullset,\{a\},X\}$.
\subsection*{83.8}
a. Let $U$ be open in $\rn$, and let $x\in U$. There is some $\ep>0$ such that the interval $(x-\ep,x+\ep)\subset U$, and since the rationals are dense, there exist $a,b\in\mathbb{Q}$ such that $x-\ep<a<x<b<x+\ep$, so $\mathscr{B}\ni(a,b)\subset U$. By lemma, $\mathscr{B}$ is a basis of the standard topology.

b. Let $\basis=\mathscr{C}$, and let $\basis'$ be the basis for the lower limit topology mentioned in the book. For any $x\in\rn$, and for any $[a,b)\in\basis$ containing $x$, we have $[a,b)\in\basis'$ containing $x$ also (with the same values of $a$ and $b$. Thus, by lemma 13.3, the lower limit topology is finer than the one generated by $\basis$. However, if we take $x=\pi$ and $B'\in\basis'$ to be $[\pi,7)$, there is no $B\in\basis$ such that $B\in B'$. If we let $B=[a,b)$ contain $\pi$, we must have $a<\pi$, which means that $B$ includes some elements not in $B'$. Thus, the lower limit topology is strictly finer, so the two topologies are distinct.
\subsection*{92.8}
Subspace of $\rn_l\times\rn$: Open sets in this product are generated by the basis consisting of elements like $[a,b)\times(c,d)$. These are rectangles with lower left corner at $a\times c$ and upper right corner at $b\times d$. These rectangles include the left edge, not including the corners. Take any nonvertical line $L$ parametrized by $t$ such that the $x$-coordinate of $L(t_1)$ is less than the $x$-coordinate of $L(t_2)$ if $t_1<t_2$. For any $t_1<t_2$, we can intersect $L$ with the rectangle $[t_1,b)\times(t_2,d)$ for appropriate values of $b,d$ to get the image of the half-open interval $[t_1,t_2)$. All intersections with the basis of the product will look like this or some open interval, so we have what is essentially the lower limit topology. For vertical lines, all intersections are open intervals, so we have the standard topology on $\rn$. 

Subspace of $\rn_l\times\rn_l$: The basis elements of this product are rectangles which include the lower and left edges, including the lower left corner. The topology on $L$ now depends more on the orientation of $L$. For horizontal or vertical lines, the answer is the same as above: the intersection of $[t_1,t_2)\times[c,d)$ with $L$ (if nonempty) is the half-open interval $[t_1,t_2)$ on some sensible parametrization of $L$ (and similarly for the vertical case). 

For lines with negative slope, they can either intersect a rectangle on their lower left half, in which case the intersection includes endpoints, or they can intersect on the upper right, in which case the intersection includes no endpoints. Thus, both the intervals $[t_1,t_2]$ and $(t_1,t_2)$ are open, as well as the closed ray $[t_2,\infty)$ from intersection with the rectangle $[t_2,\infty)\times\rn$. Thus, we see that the singleton $\{t_2\}$ is open for all values of $t_2$, which means that every subset is open, and we get the discrete topology. 

For lines with positive slope, they will always enter the rectangle on the lower or the left edge and exit through the upper or right edge. Thus, the intersections are all half-open intervals, giving us once again the lower limit topology.
\subsection*{92.9}
Let $\basis$ be the basis for the dictionary order consisting of the open intervals, and let $\basis'$ be the basis for the product of $\rn_d\times\rn$ consisting of sets of the form $\{c\}\times(a,b)$. Consider an element $x'\times y'\in(x\times y,w\times z)$ in the dictionary order. Then, the set $\{x'\}\times(y'-\ep,y'+\ep)$ for sufficiently small $\ep$ contains $x'\times y'$ and is contained in the basis element of the dictionary order topology. 

Now, consider some $c\times y\in\{c\}\times(a,b)$. We have that the interval $(c\times (y-\ep),c\times (y+\ep))$ contains $y$ and is contained in $\{c\}\times(a,b)$. Thus, by lemma 13.3, the topologies are both finer than the other one, and therefore they're the same.

This topology is strictly finer than the standard topology on $\rn^2$. The open interval $\{c\}\times(a,b)$ in the dictionary order topology is not open in $\rn^2$, but we can obtain any basis element $(a,b)\times(c,d)$ of the standard topology by taking the union $\bigcup_{a<x<b}\{x\}\times(c,d)$.
\subsection*{93.10}
The dictionary order topology on $I\times I$ is strictly finer than the product topology on $I\times I$ by the same argument as in the last part of the previous problem. The topology of $I\times I$ as a subspace of $\rn\times\rn$ on the dictionary order is strictly finer than the plain dictionary topology on $I\times I$. To see this, note that every basis element $(a\times b,c\times d)$ is the intersection of $(a\times b,c\times d)$ in $\rn^2$ (under the dictionary order) with $I^2$. However, the set $\{0.5\}\times(0.5,1]$ is open in the subspace topology (intersect $\{0.5\}\times(0.5,1.5)$ with $I^2$), but not in the dictionary order topology. Any basis element containing $0.5\times 1$ also contains elements greater than it, which the set in question does not.
\end{document}
