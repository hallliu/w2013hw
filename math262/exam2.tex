\documentclass{article}
\usepackage{geometry}
\usepackage[namelimits,sumlimits]{amsmath}
\usepackage{amssymb,amsfonts}
\usepackage{multicol}
\usepackage{mathrsfs}
\usepackage[cm]{fullpage}
\newcommand{\nc}{\newcommand}
\newcommand{\tab}{\hspace*{5em}}
\newcommand{\conj}{\overline}
\newcommand{\dd}{\partial}
\nc{\cn}{\mathbb{C}}
\nc{\rn}{\mathbb{R}}
\nc{\qn}{\mathbb{Q}}
\nc{\pd}[2]{\frac{\partial {#1}}{\partial {#2}}}
\nc{\ep}{\epsilon}
\nc{\topo}{\mathscr{T}}
\nc{\basis}{\mathscr{B}}
\nc{\nullset}{\varnothing}
\setlength{\parindent}{0mm}
\begin{document}
Name: Hall Liu

Date: \today 
\vspace{1.5cm}
\subsection*{1}
a. Since $f$ is a homomorphism of additive groups, it suffices to show that $f(x)=0$ iff $x=0$. Suppose not -- let $x\neq0$ and $f(x)=0$. Multiplying both sides by $f(1/x)$ gives $f(x/x)=f(1)=0$, which is a contradiction.

b. Suppose $s<t$. Then $t-s>0$, so we can let $u^2=t-s$. We then have $f(u^2)=f(u)^2=f(t)-f(s)$, which implies that $f(t)>f(s)$ since $f(u)^2$ is strictly positive.

c. For $n\in\mathbb{Z}$, we have $f(n)=f(1)+\cdots+f(1)=n$. If we divide this by $f(m)$ for some $m\in\mathbb{Z}^\times$, we get $f(n/m)=n/m$, so the function acts as the identity on the rationals.

d. Follows from e

e. Let $x\in\rn$. If $x\in\qn$, then $f(x)=x$. Otherwise, we have some sequence of rationals $r_n$ converging from below to $x$ and some other sequence $s_n$ converging from above to $x$. Since $f$ preserves order, we have $f(x)<f(s_n)=s_n$ for all $n$ and $f(x)>f(r_n)=r_n$ for all $n$. If $f(x)>x$, then $f(x)-x=\ep$, which is a contradiction because $s_n<x+\ep/2$ for some $n$, thus rendering $f(x)$ larger than $s_n$. Same argument for $f(x)<x$. Thus we have $f(x)=x$ for all $x\in\rn$.
\subsection*{2}
a. First, WTS that $L^c$ is open. Consider some $(a,b)$ in $A^c$, and suppose that every 'hood of $(a,b)$ intersects $L$. Let $A,B$ be 'hoods of $a,b$, resp. in $\rn_l$, so we have that $A\times B$ contains some element of $A$, or that $x=-y$ for some $y\in B$ and $x\in A$, for all neighborhoods $A$ and $B$. However, since $b\neq -a$, let us assume WLOG that $|a|<|b|$. We can find neighborhoods $[a,(a-b)/2)$ and $[b,(b-a)/2)$ that are disjoint when we take the negative of one of them. This presents a contradiction, so $L$ is closed. In addition, $L$ has the discrete topology as a subspace, since intersecting the neighborhood of any point on $L$ with the half-open rectangle with its lower left corner at that point just gives that point itself. Thus, any subset of $L$ is closed in $L$, so any subset of $L$ is closed in $\rn_l\times\rn_l$.

b,c. I couldn't find the symbol for this function, so I'm calling it $f$. We WTS that $f(A)=f(B)$ iff $A=B$, which means that for each $f(A)$, there is only one $A'$ such that $f(A')=f(A)$, which also shows injectivity. Suppose otherwise. Then, we have $U_A$ and $U_B$ such that they share rational points but there exists some point $a=(x,-x)\in A$ and $a\in L-B$. Thus, $a\in U_A$ and $a\in V_B$. Take some rational sequence converging to $A$ from the upper right. $U_A$ and $V_B$ being open, the tail of the sequence is shared between them. However, $U_A$ and $U_B$ share rational points, which implies that $U_B$ and $V_B$ are not disjoint, which is a contradiction. 

d. Cardinality of $P(L)$ is $2^\mathfrak{c}$, but the cardinality of $P(D)$ is $2^{\aleph_0}$. The former is bigger than the latter, so there can be no injection going that way.

\subsection*{3}
a. Take the preimage of $W$ under multiplication. This is an open set in $G\times G$ that includes $(e,e)$, so there is some neighborhood $X$ of $e$ such that $X\times X$ is contained in that preimage. Consider $V=X\cap X^{-1}$. This is open, since inversion is an open map. It contains $e$ because $e^{-1}=e$, and it's symmetric because if $x\in V$, then $x^{-1}\in X$ or $X^{-1}$, which means that it's in $V$ as well. Consider two elements $x$ and $y$ from $V$. We have by the definition of $X$ that $xy\in W$, so we have found a suitable neighborhood $V$.

b. Consider $xy^{-1}$. This is not equal to the identity, and it's a closed singleton. Take $G-xy^{-1}$, which is a 'hood of $e$, thus we can find some symmetric 'hood $V$ such that $VV$ does not contain $xy^{-1}$. Now suppose $Vx$ and $Vy$ share a point $a$. Then let $a=vx=wy$ with $v,w\in V$. We have $vxy^{-1}=w$ or $xy^{-1}=v^{-1}w$. Since $V$ is symmetric, this implies $xy^{-1}$ is in $VV$, contradiction. Thus $Vx$ and $Vy$ are disjoint. Futher, they're neighborhoods because they're images of the homeomorphism given by right-multiplication of $x$ and $y$, resp.

c. Repeat the same thing above, except with the set $xA^{-1}$ instead of $\{xy^{-1}\}$. $xA^{-1}$ is closed because it is the image of the composition of the inversion homeomorphism and the left-multiplication homeomorphism. We then have some symmetric 'hood $V$ such that $VV$ is disjoint from $xA^{-1}$. Suppose that $Vx$ and $VA$ share a point $c$, and let $c=vx=wa$ implying that $xa^{-1}=v^{-1}w\in VV$, which is a contradiction. $Vx$ is a 'hood of $x$ as shown above, and $VA$ is open and contains $A$ because $e\in V$ and $VA$ is the union of all the right-translated copies of $V$.

\subsection*{4}
a. Take two points $x,y$, and suppose WLOG that $d(x,A)\leq d(y,A)$. Then, $d(y,A)=\inf_{a\in A}d(y,a)\leq\inf_{a\in A}(d(x,y)+d(x,a))=d(x,y)+d(x,A)$, so $d(y,A)-d(x,A)\leq d(x,y)$, so the distance-to-A function is Lipschitz.

b. Suppose $x\in\bar{A}$. Then there is some sequence $a_n$ converging to $x$ in $A$. Since $d(a_n,x)$ can be made arbitrarily small, the distance is $0$. Suppose that $d(x,A)=0$. Then, for each $n$, let $a_n$ be such that $d(x,a_n)<1/n$. Then we have constructed a sequence in $A$ that converges to $x$, so $x\in\bar{A}$.

c. For metric spaces, compact implies sequentially compact. Construct a sequence $a_n$ such that for each $n$, $d(a_n,x)-d(A,x)<1/n$. This is possible because of the inf property. We have that some subsequence of these converges. Call this subsequence $b_n$ and its limit $b$. Suppose that $d(b,x)>d(A,x)$. Then there exists some $\ep$ such that $d(b,x)=d(A,x)+\ep\leq d(b_n,b)+d(b_n,x)$ for all $n$, or $d(b_n,b)+(d(b_n,x)-d(A,x))\geq\ep$ for all $n$. However, this is impossible, as both terms converge to $0$ as $n\to\infty$. Thus $d(b,x)=d(A,x)$ since by definition $d(A,x)\leq d(b,x)$.

\subsection*{5}
\end{document}
