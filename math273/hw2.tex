\documentclass{article}
\usepackage{geometry}
\usepackage[namelimits,sumlimits]{amsmath}
\usepackage{amssymb,amsfonts}
\usepackage{multicol}
\usepackage[cm]{fullpage}
\newcommand{\nc}{\newcommand}
\newcommand{\tab}{\hspace*{5em}}
\newcommand{\conj}{\overline}
\newcommand{\dd}{\partial}
\nc{\cn}{\mathbb{C}}
\nc{\rn}{\mathbb{R}}
\nc{\vphi}{\varphi}
\nc{\openm}{\begin{pmatrix}}
\nc{\closem}{\end{pmatrix}}
%for this hw only
\nc{\et}{e^t}
\nc{\ent}{e^{-t}}
\nc{\pd}[2]{\frac{\partial {#1}}{\partial {#2}}}
\nc{\ep}{\epsilon}
\setlength{\parindent}{0mm}
\begin{document}
Name: Hall Liu

Date: \today 
\vspace{1.5cm}

\subsection*{49.22}
Writing that equation as a system gives solutions of the form $(y,y',\ldots,y^{(n-1)})$. Arranging these as column vectors in a matrix, we get the wronskian. Theorem 2.4 says that if the wronskian is nonzero for some set of solutions $y_i$, then the $y_i$ form a fundamental matrix. (not really sure what the book wants here...)
\subsection*{49.24}
$y_1'=(e^{r_1t},re^{r_1t})'=(r_1e^{r_1t},r_1^2e^{r_1t})$. $Ay_1=(r_1e^{r_1t},-(a_2+r_1a_1)e^{r_1t})=y_1'$ by the algebraic relation between the $r_i$ and $a_i$. Thus the first column is a solution. By a similar calculation and replacing $r_1$ with $r_2$, the second column is a solution also. The solution matrix at time $0$ is
$$\begin{pmatrix}1&1\\r_1&r_2\\\end{pmatrix}$$
whose determinant is $r_1+r_2$. This is $-a_1$, which we're going to assume is nonzero. Thus the solution matrix is fundamental.
\subsection*{50.26}
We have $A(t+2\pi)Y(t+2\pi)=Y'(t+2\pi)=A(t)Y(t+2\pi)$. Similarly, $Y'(t)=A(t)Y(t)$. Rearranging and substituting to eliminate the factor of $A$, we have $Y'(t+2\pi)Y^{-1}(t+2\pi)=Y'(t)Y^{-1}(t)$. At time $0$, this is equivalent to saying $Y'(2\pi)=Y'(0)Y^{-1}(0)Y(2\pi)$, or $Y(2\pi)=Y(0)Y^{-1}(0)Y(2\pi)$. Now, we know that $Y(t+2\pi)$ is a fundamental matrix because it satisfies $Y'(t+2\pi)=A(t)Y(t+2\pi)=A(t+2\pi)Y(t+2\pi)$ and its determinant is nonzero because $\det Y$ is nonzero. Since we know that this fundamental matrix is equal to another fundamental matrix, $Y(t)Y^{-1}(0)Y(2\pi)$, at time $0$, they must be equal everywhere, so the constant matrix we're looking for is $C=Y^{-1}(0)Y(2\pi)$.
\subsection*{54.5}
Use the fundamental matrix $Y=\begin{pmatrix}t^2&t\\2t&1\\\end{pmatrix}$ given in the text. Let $$\Psi(t)=Y(t)\int_2^tY^{-1}(s)\begin{pmatrix}t^4\\t^3\end{pmatrix}ds=\openm t^2&t\\2t&1\closem\int_2^t\openm 0\\s^3\closem ds=-\openm t^2&t\\2t&1\closem\openm0\\t^4/4-4\closem=-\openm t^5/4-4t\\t^4/4-4\closem$$
This is a solution of the inhomogenous equation with initial condition $\Psi(2)=\vec{0}$. Since the fundamental matrix is valid everywhere but $0$, this solution is valid on $(0,\infty)$. 

Let $\Phi_h(t)=Y(t)Y^{-1}(2)\openm1\\4\closem=\openm-3t^2/2+3t\\-3t+3\closem$. This is a solution to the homogenous equation with initial condition as specified, so the solution to the inhomogenous equation with the specifie initial conditions is the sum of the two, or $\Phi(t)=\openm t^5/4-3t^2/2-t\\t^4/4-3t-1\closem$.
\subsection*{54.6}

Let $\vec{y}=\openm y\\y'\closem$, where $y$ is a solution of the scalar equation, so $\vec{y}'=-\openm y'\\p(t)y'+q(t)y\closem+\openm0\\f(t)\closem$. If we let $A=\openm0&1\\q(t)&p(t)\closem$, then we have $A\vec{y}=\vec{f}(t)-\vec{y'}$ or $\vec{y'}=\vec{f}(t)-A\vec{y}$, where $\vec{f}(t)=\openm0\\f(t)\closem$.

Since the two solutions $\varphi_1$ and $\varphi_2$ are independent, so are the columns of the matrix $\Phi(t)$. In addition, since the $\varphi$s are solutions of the homogenous scalar equation, the columns are also solutions of the homogenous system obtained by letting $\vec{f}(t)=0$. 

b. The determinant of $\Phi$ is $\vphi_1\vphi_2'-\vphi_1'\vphi_2$, so the inverse is $\frac{1}{|\Phi|}\openm\vphi_2'&-\vphi_2\\-\vphi_1'&\vphi_1\closem$. Multiplying this by $\openm0\\f(t)\closem$ gives $\frac{1}{|\Phi|}\openm-\vphi_2f\\\vphi_1f\closem$. Integrate this from $t_0$ to $t$ and multiply by $\Phi(t)$ to get the solution $\Psi$ (no explicit formula available).

c. The first component of the above is the equation presented in the book, which can be obtained by moving the matrix $\Phi(t)$ inside the integral and evaluating the first component. It and its derivative are both zero at $t_0$ because $\psi_1'=\psi_2$, and they're both components of $\Psi$, which is $0$ at $t_0$.
\subsection*{59.8}
We evaluate $e^{At}=e^{-2It}e^{Ut}$, where $U=\openm0&1&0\\0&0&1\\0&0&0\closem$. The first exponential is just a diagonal matrix with all nonzero entries $e^{-2t}$. We have $U^2=\openm0&0&1\\0&0&0\\0&0&0\closem$, and this is the last nonzero power of $U$. Thus,
$$e^{Ut}=\left(I_3+Ut+U^2t^2/2\right)=\openm1&t&t^2/2\\0&1&t\\0&0&1\closem$$
so
$$e^{At}=e^{-2t}\openm1&t&t^2/2\\0&1&t\\0&0&1\closem$$

Derivative of $e^{At}$ is 
$$\openm-2e^{-2t}&e^{-2t}-2te^{-2t}&te^{-2t}-t^2e^{-2t}\\0&e^{-2t}&e^{-2t}-2te^{-2t}\\0&0&e^{-2t}\closem$$
and
$$Ae^{At}=e^{2t}\openm-2&1&0\\0&-2&1\\0&0&-2\closem\openm1&t&t^2/2\\0&1&t\\0&0&1\closem=e^{-2t}\openm-2&-2t+1&-t^2+t\\0&-2&-2t+1\\0&0&-2\closem$$ which is equal to the derivative.
\subsection*{70.13}
The characteristic polynomial is $(1-x)(3-x)-8=0$ or $x^2-4x-5=0$, which has solutions $5$ and $-1$. The eigenvectors corresponding to 5 can be found by $\openm1&2\\4&3\closem\openm a\\b\closem=\openm5a\\5b\closem$. We get $\openm1\\2\closem$ here, and solving a similar equation for $-1$ gives $\openm1\\-1\closem$. Then, two independent solutions are $\openm e^{5t}\\2e^{5t}\closem$ and $\openm e^{-t}\\-e^{-t}\closem$. The fundamental matrix is then obtained by putting these two together. To find the particular solution with given initial conditions, we take the inverse of the fundamental matrix and evaluate it at zero, multiply it by $\openm3\\3\closem$, then multiply the fundamental matrix by that. This is $\openm2e^{5t}+e^{-t}\\4e^{5t}-e^{-t}\closem$
\subsection*{71.14}
Char. poly is $(2-x)(-4-x)+9=0$ which has a double root at $-1$. Trying to find eigenvectors gets us only the vector $\openm1\\1\closem$, which doesn't form a basis. Then, we want to look at the space $X_1$ spanned by solutions of $(A+I)^2\vec{x}=0$. Since we expect this to be a 2-dimensional space, it actually spans all of $\rn^2$, so we have $\eta\in X_1$, and the solution we want is $e^{tA}\eta=e^{-t}e^{(A+I)t}\eta$. Expanding out the matrix exponential gets us $I+(A+I)t+\cdots$, with the tail of the series omitted because it's sent to zero upon multiplying by $\eta$. Thus, the solution is $$e^{-t}\eta+te^{-t}\openm3&-3\\3&-3\closem\eta=\openm e^{-t}-3te^{-t}\\2e^{-t}-3te^{-t}\closem$$
To make a fundamental matrix, just replace $\eta$ with something independent of what we're given. Use $\openm1\\0\closem$, and we get another solution $\openm e^{-t}+3te^{-t}\\3te^{-t}\closem$, and we get the fundamental matrix $e^{-t}\openm1-3t&1+3t\\2-3t&3t\closem$
\subsection*{71.15}
Char polynomial is $(1-x)((1-x)(-1-x)+1)+3(8-5(1-x))=-x^3+x^2+15x+9$ from expanding along the first row. This has $\lambda_1=-3$ as a root, leaving behind the polynomial $-x^2+4x+3$ which has roots $\lambda_2,\lambda_3=2\pm\sqrt{7}$, so the matrix has $3$ distinct eigenvalues. Omitting the work for finding eigenvectors because it's a pain in the ass to type up. Eigenvector for $-3$ is $\openm3\\-7\\-4\closem$, eigenvector for $2+\sqrt{7}$ is $\openm-3/2+\lambda_2/2\\17/2-3\lambda_2/2\\1\closem$, and eigenvector for $2-\sqrt{7}$ is $\openm-3/2+\lambda_3/2\\17/2-3\lambda_3/2\\1\closem$. This makes the fundamental matrix 
\begin{equation*}
\openm 
3e^{-3t}&(-3/2+\lambda_2/2)e^{\lambda_2t}&(-3/2+\lambda_3/2)e^{\lambda_3t}\\
-7e^{-3t}&(17/2-3\lambda_2/2)e^{\lambda_2t}&(17/2-3\lambda_3/2)e^{\lambda_3t}\\
-4e^{-3t}&e^{\lambda_2t}&e^{\lambda_3t}
\closem
\end{equation*}
The actual solution is really god-awful. Trust me on its correctness. I checked in Mathematica.
\large
$$\openm\frac{13}{e^{3 t} 3}-\frac{13}{6} e^{\left(2-\sqrt{7}\right) t}-\frac{e^{\left(2-\sqrt{7}\right) t}}{3 \sqrt{7}}-\frac{13}{6} e^{\left(2+\sqrt{7}\right) t}+\frac{e^{\left(2+\sqrt{7}\right) t}}{3 \sqrt{7}}\\-\frac{91}{9 e^{3 t}}+\frac{73}{18} e^{\left(2-\sqrt{7}\right) t}+\frac{187 e^{\left(2-\sqrt{7}\right) t}}{9 \sqrt{7}}+\frac{73}{18} e^{\left(2+\sqrt{7}\right) t}-\frac{187 e^{\left(2+\sqrt{7}\right) t}}{9 \sqrt{7}}\\-\frac{52}{9 e^{3 t}}-\frac{11}{18} e^{\left(2-\sqrt{7}\right) t}+\frac{89 e^{\left(2-\sqrt{7}\right) t}}{18 \sqrt{7}}-\frac{11}{18} e^{\left(2+\sqrt{7}\right) t}-\frac{89 e^{\left(2+\sqrt{7}\right) t}}{18 \sqrt{7}}\closem$$
\normalsize
\subsection*{71.16}
Well isn't that nice. They gave us the characteristic polynomial. Too bad it has multiple roots. I'm going to use row and column vectors interchangably due to ease of notation. Should be clear from context what everything is. Consider eigenvectors with eigenvalue $1$, or the kernel of $A-I$. Fiddling with the system of equations gives that this eigenspace is one-dimensional, so we have to take the long way around and consider the kernels of $(A-I)^2$ and $(A+I)^2$. The first is the matrix 
$$\left(
\begin{array}{cccc}
 -4 & 4 & 8 & -8 \\
 0 & -4 & 0 & 8 \\
 -4 & 4 & 8 & -8 \\
 0 & -4 & 0 & 8
\end{array}
\right)$$
This looks pretty singular. Fixing the first and second components will determine the rest of them in the kernel. Thus, let one basis vector be $b_1=(2,0,1,0)$ and the other one be $b_2(0,2,0,1)$.

The next matrix, $(A+I)^2$ is 
$$\left(
\begin{array}{cccc}
 8 & 0 & -8 & 0 \\
 8 & 8 & -8 & -8 \\
 4 & 0 & -4 & 0 \\
 4 & 4 & -4 & -4
\end{array}
\right)$$

Same setup as above: first two components determine the rest of it. Two nice basis vectors are $b_3=(1,0,1,0)$ and $b_4=(0,1,0,1)$. 

Now write the conventional basis vectors in terms of these. We have $e_1=(2,0,1,0)-(1,0,1,0)$, $e_2=(0,2,0,1)-(0,1,0,1)$, $e_3=(2,0,2,0)-(2,0,1,0)$, and $e_4=(0,2,0,2)-(0,2,0,1)$. The solutions corresponding to these initial conditions are:

$e_1$: $e^t(I+t(A-I))b_1-e^{-t}(I+t(A+I))b_3=e^t(2,2t,1,t)-e^{-t}(1,0,1,0)$\\
$e_2$: $e^t(I+t(A-I))b_2-e^{-t}(I+t(A+I))b_4=e^t(0,2,0,1)-e^{-t}(t,1,t,1)$\\
$e_3$: $2e^t(I+t(A-I))b_3-e^{-t}(I+t(A+I))b_1=e^{-t}(2,0,2,0)-e^t(2,2t,1,t)$\\
$e_4$: $2e^t(I+t(A-I))b_4-e^{-t}(I+t(A+I))b_2=2e^{-t}(t,1,t,1)-e^t(0,2,0,1)$

The fundamental matrix is 
$$\openm
2\et-\ent & -t\ent & 2\ent-2\et & 2t\ent\\
2t\et & 2\et-\ent & -2t\et & 2\ent-2\et\\
\et-\ent & -t\ent & 2\ent-\et & 2t\ent\\
t\et & \et-\ent & -t\et & 2\ent-\et\\
\closem$$
The nice thing about this fundamental matrix is that its initial value is the identity, so all we have to do to get the particular solution is to multiply by $\eta$, so we get $(4\et-3\ent,4t\et,2\et-3\ent,2t\et)$. This works, I think.

\end{document}
