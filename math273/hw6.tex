\documentclass{article}
\usepackage{geometry}
\usepackage[namelimits,sumlimits]{amsmath}
\usepackage{amssymb,amsfonts}
\usepackage{multicol}
\usepackage[cm]{fullpage}
\newcommand{\nc}{\newcommand}
\newcommand{\tab}{\hspace*{5em}}
\newcommand{\conj}{\overline}
\newcommand{\dd}{\partial}
\nc{\cn}{\mathbb{C}}
\nc{\rn}{\mathbb{R}}
\nc{\vphi}{\varphi}
\nc{\openm}{\begin{pmatrix}}
\nc{\closem}{\end{pmatrix}}
\nc{\pd}[2]{\frac{\partial {#1}}{\partial {#2}}}
\nc{\ep}{\epsilon}
\setlength{\parindent}{0mm}
\DeclareMathOperator{\tr}{tr}
\begin{document}
Name: Hall Liu

Date: \today 
\vspace{1.5cm}

\subsection*{4.3.3}
Matrix is $\openm0&1\\-1&0\closem$. Eigenvalues are $\pm i$. Since these have real part zero and are simple, zero is stable.
\subsection*{4.3.5}
Matrix is $\openm0&1\\-1&-2\closem$. Single eigenvalue $-1$ with multiplicity $2$. It has negative real part so zero is asm. stable.
\subsection*{4.3.9}
This has a single positive eigenvalue at $1/2(\sqrt{17}-1)$, so zero is not stable. 
\subsection*{4.3.13}
a. $D_1=p$ and $D_2=pq$. We want these to be strictly positive, and that only happens if both $p$ and $q$ are strictly positive.

b. $D_1=p$ and $D_2=pq-r$ and $D_3=\det\openm p&r&0\\1&q&0\\0&p&r\closem=pqr-r^2$. We need $p>0$ and $pq>r$ and $r(pq-r)>0$ which implies that we also need $r>0$.

c. $D_1=p$ and $D_2=pq-r$ and $D_3=\det\openm p&r&0\\1&q&s\\0&p&r\closem=pqr-r^2-p^2s$ and 
$$D_4=\det\openm p&r&0&0\\1&q&s&0\\0&p&r&0\\0&1&q&s\closem=pqrs-r^2s-p^2s=s(pqr-r^2-p^2s)$$
Then we need $p>0$ and $pq>r$ and $r(pq-r)>p^2s$ and $s>0$. Under the rest of these, $pq>r$ is equivalent to $q>0$, since if $q\leq0$ we'd have $r\leq0$ which means $p^2s\leq0$ which means $s\leq0$ which is a contradiction, and if $pq\leq r$ we'd have $r>0$ which means $p^2s<0$ which is a contradiction.
\subsection*{4.3.15}
\end{document}
