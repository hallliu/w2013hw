\documentclass{article}
\usepackage{geometry}
\usepackage[namelimits,sumlimits]{amsmath}
\usepackage{amssymb,amsfonts}
\usepackage{multicol}
\usepackage[cm]{fullpage}
\newcommand{\nc}{\newcommand}
\newcommand{\tab}{\hspace*{5em}}
\newcommand{\conj}{\overline}
\newcommand{\dd}{\partial}
\nc{\cn}{\mathbb{C}}
\nc{\rn}{\mathbb{R}}
\nc{\vphi}{\varphi}
\nc{\openm}{\begin{pmatrix}}
\nc{\closem}{\end{pmatrix}}
\nc{\pd}[2]{\frac{\partial {#1}}{\partial {#2}}}
\nc{\ep}{\epsilon}
\nc{\ddt}{\frac{d}{dt}}
\setlength{\parindent}{0mm}
\DeclareMathOperator{\tr}{tr}
\begin{document}
Name: Hall Liu

Date: \today 
\vspace{1.5cm}

\subsection*{3.1.4}
First, we have 
$$\frac{d}{dt}\int_t^\infty H(t,s)ds=-\ddt\lim_{a\to\infty}\int_a^tH(t,s)ds=-H(t,t)+\int_t^\infty\pd{H}{t}(t,s)ds$$
by the chain rule. Using this, we have 
$$y'(t)=ie^{it}+\alpha\left(\int_t^\infty\cos(t-s)\frac{y(s)}{s^2}ds\right)$$
since $\sin(t-t)=0$. We also make the assumption that $t$ stays away from zero, as otherwise the integral blows up. Differentiating again we have
$$y''(t)=-e^{it}-\alpha\left(\frac{y(t)}{t^2}+\int_t^\infty\sin(t-s)\frac{y(s)}{s^2}ds\right)$$
Pulling the integral out and combining with $-e^{-it}$, we have 
$$y''(t)=-y(t)-\alpha\frac{y(t)}{t^2}\implies y''(t)+(1+\alpha/t^2)=0$$
\subsection*{3.1.5}
The associated integral equation is 
$$y(t)=2+\int_0^t-y(s)ds$$
so we have $\phi_0=2$, $\phi_1=2+\int_0^t-2ds=2-2t$,$\phi_2=2-\int_0^t2-2sds=2-2t+t^2$, and in general,
$$\phi_n(t)=\sum_{i=0}^n\frac{2t^i(-1)^i}{i!}$$
which converges to $2e^{-t}$ as $n\to\infty$. This is a solution.
\subsection*{3.1.7}
$$|f(t,y_1)-f(t,y_2)|=|t|y_1|-t|y_2||\leq |t|||y_1|-|y_2||\leq |y_1-y_2|$$
\subsection*{3.1.9}
$$\phi_n(t)=\sum_{i=0}^n\frac{t^i}{i!}$$
which approaches $e^t$.
\subsection*{3.1.12}
a. Following proof of lemma 3.2, we have that $\phi_0$ is defined for $|t|\leq\alpha$ since it's constant at $y_0$. Assume by induction that $\phi_n$ is defined there. Then,
$$|\phi_{n+1}-y_0|=\left|z_0t-\int_0^t(t-s)g(s,y(s))ds\right|\leq |z_0t|+\left|\int_0^t(t-s)g(s,y(s))ds\right|\leq|z_0t|+M\int_0^t|t-s|ds\leq|z_0|\alpha+M\alpha^2/2$$
Plugging in values for $\alpha$ that were given, we have that the above expression is bounded above by
$$\frac{|z_0|b}{|z_0|+Ma/2}+\frac{Mab/2}{|z_0|+Ma/2}=b$$
so $\phi_{n+1}$ stays inside $R$.

b. We have 
$$r_j(t)=|\phi_j(t)-\phi_{j+1}(t)|=\left|\int_0^t(t-s)(g(s,\phi_{j-1})-g(s,\phi_{j}))\right|\leq\int_0^t|t-s||g(s,\phi_{j-1})-g(s,\phi_{j})|\leq K\int_0^t|t-s|r_{j-1}(s)$$
the last step by the boundedness of the first derivative. For $j=0$, we have
$$r_0(t)=\left|z_0t-\int_0^t(t-s)g(s,y_0)\right|\leq |z_0t|+Mt^2/2$$
We hypothesize that 
$$r_j(t)\leq K^{j-1}\left(\frac{|z_0||t|^{2j+1}}{(2j+1)!}+\frac{M|t|^{2j+2}}{(2j+2)!}\right)$$
It's obviously true for $j=0$, so assume true for $j=p-1$ and we have
\begin{align*}
r_p(t)&\leq K\int_0^t(t-s)r_{p-1}\leq K^{p-1}\int_0^t(t-s)\left(\frac{|z_0|s^{2p-1}}{(2p-1)!}+\frac{Ms^{2p}}{(2p)!}\right)\\
&=K^{p-1}\int_0^t\left(\frac{|z_0|s^{2p-1}t}{(2p-1)!}+\frac{Ms^{2p}t}{(2p)!}-\frac{|z_0|s^{2p}}{(2p-1)!}-\frac{Ms^{2p+1}}{(2p)!}\right)\\
&=K^{p-1}\left(\frac{|z_0|s^{2p+1}}{(2p)!}+\frac{Ms^{2p+2}}{(2p+1)!}-\frac{|z_0|s^{2p+1}}{(2p-1)!(2p+1)}-\frac{Ms^{2p+2}}{(2p)!(2p+2)}\right)\\
&=K^{p-1}\left(\frac{|z_0|t^{2p+1}}{(2p+1)!}+\frac{Mt^{2p+2}}{(2p+2)!}\right)
\end{align*}
in the case where $t\geq 0$, and the case where $t<0$ is similar. Now, since we have $|t|<\alpha$, the sum of all the $r_p$ is bounded above by some big constant times $\sum\frac{c^n}{n!}$, so we have the uniform convergence of the sequence of $\phi$s.

Call the limit $\phi(t)$. The argument in the book for continuity of $\phi$ works here too since all it depended on was that $\phi_j$ converges uniformly to $\phi$. To show that it satisfies the integral equation, we just have to show that 
$$\phi(t)=\lim_{n\to\infty}\phi_n(t)=y_0+z_0t-\lim_{n\to\infty}\int_0^t(t-s)g(s,\phi_{n-1}(t))=y_0+z_0t-\int_0^t(t-s)g(s,\phi(t))$$
which essentially boils down to showing that the limit of the integral equals the other integral. Then, we have
$$\left|\int_0^t(t-s)(g(s,\phi_j(s))-g(s,\phi(s)))\right|\leq K\left|\int_0^t|t-s||\phi(s)-\phi_j(s)|\right|$$
which we can see goes to $0$ as $j\to\infty$ since the difference of the $\phi$s goes to $0$ and the $t-s$ term is bounded by $\alpha$.
\subsection*{3.1.13}
later
\subsection*{3.1.14}
Each of the successive approximations is $0$, since $f(s,0)=0$ in this case. They converge to a solution, but a trivial one.
\subsection*{3.3.2}
$f$ is defined on any closed rectangle about $(0,0)$ as long as the rectangle is less than $2$ wide. For $t\leq0$ on this rectangle, $f$ is constant in $y$, so it's nondecreasing. Fix some $t>0$. Then, as $y$ increases from $-\infty$, $f$ is constant at $2t$ when $y<0$, and is monotone decreasoing from $y=0$ to $y=t^2$, all the while remaining below $2t$. At $y=t^2$, $f=-2t$, and it stays at this value as $y$ further increases. Thus the hypotheses of the theorem are satisfied and we have uniqueness.
\subsection*{3.4.3}
\end{document}
