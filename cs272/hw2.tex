\documentclass{article}
\usepackage{geometry}
\usepackage[namelimits,sumlimits]{amsmath}
\usepackage{amssymb,amsfonts}
\usepackage{multicol}
\usepackage{mathrsfs}
\usepackage{graphicsx}
\usepackage[cm]{fullpage}
\newcommand{\nc}{\newcommand}
\newcommand{\tab}{\hspace*{5em}}
\newcommand{\conj}{\overline}
\newcommand{\dd}{\partial}
\nc{\nn}{\mathbb{N}}
\nc{\pd}[2]{\frac{\partial {#1}}{\partial {#2}}}
\nc{\ep}{\epsilon}
\nc{\nullset}{\varnothing}
\setlength{\parindent}{0mm}
\begin{document}
Name: Hall Liu

Date: \today 
\vspace{1.5cm}
\subsection*{4.2}
a. True. Prim's algorithm only cares about the ordering of the costs of the edges. Since squaring preserves this order, Prim's algorithm will choose the exact same edge on the new costs as on the old costs, so the same spanning tree will be produced.

b. False. Consider the following graph.

\includegraphics[height=150pt]{scripts_2/4_2b.png}

As is, the shortest path from $v_1$ to $v_2$ follows the edge with cost $2$. However, upon squaring the costs, the shortest path becomes the one along the three edges of cost $1$ each.
\subsection*{4.5}
Let the houses be labeled $h_1,\ldots,h_l$ from left to right and spaced at arbitrary distances. Let the towers produced by the below algorithm be labeled $g_1,\ldots,g_n$ from left to right.
\begin{verbatim}
Place t_1 4 miles to right of first house.
let i=1
while there exist houses more than 4 miles from a tower:
    place t_(i+1) 4 miles to the right of the leftmost uncovered house
    i++
end
\end{verbatim}
It is fairly easy to see that this algorithm will produce a cover of the houses. If there exists some uncovered house somewhere, the algorithm would have placed a cell tower 4 miles to its right, thus covering it. To prove optimality, let a set of optimal towers be denoted $s_1,\ldots,s_m$ from left to right. We want to show that $n\leq m$. Further, let the $h_i,s_i,g_i$ take positive real values representing their position.

First, we note that $s_1\leq t_1$. If $s_1>t_1$, then $s_1-h_1>t_1-h_1=4$, which leaves the first house uncovered and therefore is a contradiction. Next, assume by induction that $s_{k-1}\leq t_{k-1}$. Let $h_i$ be the lowest house which is not covered by $s_1,\ldots,s_{k-1}$ and let $h_{i'}$ be the corresponding house for the $t$s. We must have $h_i\leq h_{i'}$, since $s_{k-1}$ has less reach than $t_{k-1}$. Now consider the placement of $s_k$ and $t_k$. We must have $s_k-h_i\leq 4$, since $h_i$ needs to be covered by the next tower. However, from our description of the greedy algorithm, we know that $t_k=h_{i'}+4\geq h_i+4\geq s_k$. Thus, $s_k\leq t_k$ for all $k$.

Suppose now that $m<n$. Then, we have $s_m\leq t_m$. Since $t_{m+1}$ exists, we know that there must still be uncovered houses to the right of $t_m+4$, or to the right of $s_m+4$. This is a contradiction, so we must have $n\leq m$, therefore showing optimality.
\subsection*{4.9}
a. No. Consider the following graph

\includegraphics[width=0.4\textwidth]{scripts_2/4_9a.png}

A min. bottleneck tree could include the edges with costs 5,4,3, but this is not a min. spanning tree which would include the edges with costs 5,3,2.

b. Yes. First we do 4.8 so we can use the uniqueness property. Note that in each step in Prim's algorithm, the edge that is added is the min. cost edge from $S$ to $V\backslash S$. By the cut property, this edge is included in every min. spanning tree. Thus, all the edges in the result of Prim's algorithm are part of every min. spanning tree, and therefore the unique  min. spanning tree.

Consider the reverse-deletion algorithm. At some point, we come to an edge $e$ that is not in a cycle. Then, $e$ must be part of any min. bottleneck tree, since removing $e$ would disconnect the graph and the only way to reconnect it is to include one of the more expensive edges. We then have that any spanning tree which includes $e$ and no edges costlier than $e$ is a min. bottleneck tree. In particular, the min. spanning tree which is the result of the reverse deletion algorithm is a min. bottleneck tree. Since the min. spanning tree is unique, all min.spanning trees are min. bottleneck trees.
\subsection*{4.10}
First note that if $T$ is no longer a min. spanning tree after adding the additional edge $e$, then any new min. spanning tree $T'$ must include the edge $e$ because otherwise $T$ would have not been a min. spanning tree in the first place since $T'$ on the original graph has a lower cost.

a. Denote $G$ with $e$ added as $G'$. Assume that the tree is represented as an adjacency list. Add edge $e$ to the tree. We are now guaranteed to have exactly one cycle in the result, $T'$, which includes $e$. Run DFS (the version on p84 of the text) on $T'$ starting from one of the endpoints $v$ of $e$ and along $e$, with the modification that we keep track of the origin of each explored vertex. Stop when we reach $v$ again. This is guaranteed to happen at some point because $v$ is part of a cycle, and tracing back the origin of $v$ when we reach it the second time will produce that cycle. This runs in $O(|V|)$ time, since $T'$ has $|V|$ edges and $|V|$ vertices and DFS runs in $|V|+|E|$ time. Find the edge with the maximum cost in the cycle. This again takes $O(|V|)$ time at worst because the cycle contains at most $|V|$ edges. If this costliest edge is not $e$, then $T$ cannot be a min. spanning tree for $G'$ because it contains the costliest edge in a cycle. If this costliest edge is $e$, then $T$ remains a min. spanning tree for $G'$ because any min. spanning tree of $G'$ cannot contain $e$.

b. After running through the process in part (a), simply remove the costliest edge in the cycle to obtain a new min. spanning tree. To show that this is a min. spanning tree, consider the reverse-deletion algorithm on $G'$. Classify the edges of $G'$ as the following: $e$, that edge that was added, edges that are not in $T'$, edges that are in $T'$ but not in the cycle formed by $e$, and edges in the cycle formed by $e$. In addition, denote the costliest edge in the cycle by $e_m$. 

For the edges that the algorithm considers before $e_m$, those edges which are not in $T'$ will be removed because they are still in a cycle. Those edges which are in $T'$ but not in the cycle created by $e$ are still not removed. These edges are in $T'$ originally because their removal would have caused a disconnection when running the reverse deletion algorithm on $G$. Here, their removal will still cause a disconnection. To see this, denote the state of $G$ at the consideration of this edge as $F$ and the state of $G'$ as $F'$, and call the edge in question $e'$. By induction, assume that the only difference between $F$ and $F'$ is the addition of $e$ in $F'$. Suppose that $e'$ is a part of a cycle in $F'$ but not in $F$. Then, the cycle in $F'$ contains $e$, which implies that $e'$ is actually a part of the cycle in $T'$. 

When we get to $e_m$, it is removed as the costliest edge in a cycle. Proceeding further, the same argument as above shows that the sequence of edges that are removed/kept is the same except for $e$, which is included into the new min. spanning tree. Thus, the replacement of $e_m$ by $e$ yields a min. spanning tree for $G'$.

\subsection*{4.15}
Construct a graph $G$ with vertices labeled from $1$ to $n$, where vertices $i$ and $j$ are connected by an edge if students $i$ and $j$ have overlapping shifts. While there exist vertices that are unsupervised, find the vertex not already in the supervising set with the highest number of unsupervised neighbors and add it to the supervising set, then mark all it and all its immediate neighbors as supervised.  Repeat until done.

The algorithm terminates because at least one vertex is marked as supervised on each iteration, and there are a finite number of vertices. To prove optimality, we will show that after the $i$th iteration, the number of students who are supervised is the greatest number that can be supervised with $i$ supervisors. Do this by induction. After the first iteration, since we found the vertex with the highest degree, no other vertex can supervise as many other ones. Now, assume that it's true for the $i-1$th iteration. Since we are picking the vertex which will supervise the highest number of unsupervised vertices on the $i$th iteration, the assertion is still true after the $i$th iteration. 

Now, suppose that our algorithm produces a supervising set of size $m$, and there exists another supervising set of size $m'<m$. Then, by the assertion proved above, the greatest number of vertices that can be supervised by $m'$ supervisors is the number supervised after the $m'$th iteration, which is less than $n$, so contradiction.

\end{document}
