\documentclass{article}
\usepackage{geometry}
\usepackage[namelimits,sumlimits]{amsmath}
\usepackage{amssymb,amsfonts}
\usepackage{multicol}
\usepackage{mathrsfs}
\usepackage{graphicx}
\usepackage{alltt}
\usepackage[cm]{fullpage}
\newcommand{\nc}{\newcommand}
\newcommand{\tab}{\hspace*{5em}}
\newcommand{\conj}{\overline}
\newcommand{\dd}{\partial}
\nc{\nn}{\mathbb{N}}
\nc{\pd}[2]{\frac{\partial {#1}}{\partial {#2}}}
\nc{\ep}{\epsilon}
\nc{\nullset}{\varnothing}
\DeclareMathOperator{\opt}{opt}
\begin{document}
Name: Hall Liu

Date: \today 

\subsection*{6.8}
a. Let $3$ robots arrive on day $1$ and $100$ arrive on day $2$, and suppose the EMP destroys $99$ robots after $1$ day of charging and $101$ after two days. Then, the optimal solution is to detonate on both days, destroying $3$ on day $1$ and $99$ on day $2$ for a total of $102$. The algorithm would decide to detonate only on day $2$, only destroying $100$.

\noindent b. The solution should follow a similar flavor to 6.12, in that the "splitting" of the problem is accomplished in much the same way. That is, the 'event' should always occur on the last element in the sequence, and given an optimal solution, if we pick the next-to-last element of the sequence on which the event occurs and split the sequence immediately after that element, then both resulting subsequences will have an optimal arrangement of events. Thus, we have the property that any optimal arrangement of events will be composed of two optimal subarrangements.

Let $A_i$ denote an optimal schedule of EMPs on the first $i$ days and let $d(A_i)$ be the number of robots destroyed by the arrangement. Then, by the argument above, for any $i$, $A_i=A_j\cup\{i\}$ for some $0\leq j<i$. Define $A_0=\nullset$ and $d(A_0)=0$. We have that $A_1=\{1\}$ and $d(A_1)=\min(f(1),x_1)$. Then, looping from $i=2$ to $n$, we calculate $d(A_i)$ by taking $\displaystyle\max_{0\leq j<i}\{d(A_j)+\min(f(i-j),x_i)\}$ and $A_i$ by taking $A_j\cup\{i\}$ for the $j$ at which the aforementioned maximum occurs.
\subsection*{6.15}
a. Let the sequence of coordinates be $-1,1,2,-1,1,-1,1$. Then, the $6$th event is illegal, and the algorithm returns the events $1,4,7$. However, we can do better by observing events $2,3,5,7$. 

\noindent b. We can do a slight modification on the principle described in the last problem. While this problem doesn't have the property that the splitting an optimal arrangement results in two more optimal arrangements, we at least have that splitting provides an optimal arrangement in the first subsequence. Thus, we can still split at the last event and optimize over that. If we let $c_i$ denote the coordinate of the $i$th event and keep the other notation, we have that $\displaystyle|A_i|=\max_{j\in L_i}\{|A_j|+1\}$, where $L_i$ is the set of `legal' elements ($L_i=\{l|l\in[0,i-1],|c_l-c_j|\leq l-j\}$). 
\subsection*{6.17}
a. Let the sequence of prices be $1,2,5,3,4$. The algorithm would take days $1,2,3$ and return $3$, whereas the correct maximum length is $4$ given by days $1,2,4,5$.

\noindent b. This problem splits in much the same way as the previous ones, except it's reversed in that the first element of a sequence is required to be part of the subsequence. If we took a rising trend of maximum length $n_j$ and looked at it starting at the second day, we'd have that the sub-subsequence is a rising trend of maximum length on the days from $n_2$ to the end, as otherwise we'd be able to rejigger it so that the original rising trend is no longer maximum. Let $A_i$ be the maximum rising trend from day $i$ to day $n$, so we have $A_n=\{n\}$ and $l(A_n)=1$. Then, for $i=n-1$ to $1$, let $\displaystyle l(A_i)=1+\max_{j\in \Omega}\{l(A_j)\}$, where $\Omega=\{j>i|P[j]>P[i]\}$. This takes the length of the longest maximum rising trend beginning after $i$ which is `compatible' with $P[i]$ and tacks on one more.
\subsection*{6.20}

Since order/contiguity of the hours spent on a project doesn't matter as to the grade, assume that the $H$ hours are divided into $n$ sections each of length $h_i$ being spent on course $i$. Assuming we have an optimal schedule, if we remove $h_n$ from the list, we get an optimal schedule on $n-1$ projects for $H-h_n$ hours. Note that for any given number of projects, we just need to maximize the sum of the project grades. 

Define the value of a particular schedule as its total grade, and let $s(r,h)$ be the optimal schedule with the first $r$ projects in $h$ hours. For all $h\leq H$, we have $s(1,h)=f_1(h)$. Then, we can compute $s(i,h)$ iteratively (assuming values of $s(i-1,h')$ are known for $h'\leq H$) by taking $\displaystyle v(s(i,h))=\max_{h'\leq h}\{v(s(i-1,h'))+f_i(h-h')\}$ and then recording the appropriate schedule in $s(i,h)$. The answer comes out when we compute $s(n,H)$. There are $nH$ schedules we need to compute, and each schedule takes at most $H$ steps of constant time, so the running time is $O(nH^2)$.
\subsection*{6.27}
First, assume that $g_1>0$, since otherwise we solve the problem for $n-1$ days starting on day $2$. To resolve ambiguities in the problem statement, assume that any gas ordered arrives in the beginning of the day that it's ordered on and that the storage cost is only charged gas left in the tank overnight and not for gas sold during the day.

Generalize the problem slightly so we can start with some $l<L$ gallons in the tank. Then, if we omit day $1$ from the optimal order schedule with $l$ gallons left, the resulting order schedule is optimal for the $n-1$ remaining days, but with some $l'<L$ gallons left in the tank. Denoting the cost of the optimal subschedule as $C$, we have that the cost of the whole schedule is $C+l'c+\delta P$, where $\delta=1$ iff we order on the first day (in the style of the Kronecker delta). Thus, since every optimal schedule can be divided up this way, we can just minimize over the value $C+l'c+\delta P$.

Denote $O[i,l]$ as the optimal order schedule starting on day $i$ with $l$ gallons left in the tank, and let $c(O[i,l])$ denote its cost. We have $c(O[n,l])=P$ for $l<g_n$, $0$ for $l=g_n$, and infinite otherwise, since we wish to respect the constraint that the tank be empty at the end of the $n$th day. For the $i$th day, we let $\displaystyle c(O[i,l])=\min_{k\leq q\leq L-l}\{c(O[i+1,l+q-g_i])+(l+q-g_i)c+\delta P\}$, where $k=\max(g_i-l,0)$, $q$ is the amount to order on day $i$, and $\delta=0$ iff $q=0$. This encompasses all the things we are allowed to do on day $i$, and since we are guaranteed that an optimal order scheduling will have an optimal subscheduling starting on the day after with some initial capacity, we have that this is optimal for day $i$ and initial volume $l$. Note that the condition of the empty tank on the last day gets preserved, as the infinity propogates itself down in an appropriate manner. For each day, we calculate $L$ values, each of which takes at most $L$ constant-time operations. Thus, the running time is $O(nL^2)$.
\subsection*{7.5}
No. Consider the two cuts on the following graph: one through all the edges of length $1$ and one through the two edges of length $3$. In the original graph, the first cut is the min cut, since it has capacity $5$. However, adding one to every edge makes the capacity of the first cut $10$, but the second cut now only has capacity $8$, so it becomes the min cut.

\includegraphics[width=0.3\textwidth]{scripts_5/7_5_base.png}
\subsection*{7.9}
This can be reduced to the circulation problem described in 7.7. Let each person be a source of capacity $1$ and let each hospital be a sink of capacity $\lceil n/k\rceil$. Have a directed edge connecting a person to a hospital if the person is within half a mile of the hospital. Then, if the circulation problem has a solution, each person will have a flow of value $1$ to some hospital (indicating the one that they should go to) and each hospital has no more than $\lceil n/k\rceil$ going into them. Conversely, if this problem has a solution, it translates directly into a solution for the circulation problem for this graph in the same manner.
\subsection*{7.13}
We reduce the node-capacitated graph to a non-node-capacitated graph by separating each node into two nodes, the in node and the out node. Let all the in edges to the original node connect to the in node, and let all the out edges from the original node go out from the out node. Connect the in node to the out node with an edge of the same capacity as the node capacity and call it the inner edge. A valid flow on the derived graph translates into a valid flow on the edge-limited graph. Capacity conditions on the edges are preserved because the flow through edges present in the edge-limited graph are the same as the corresponding edges in the derived graph. Capacity conditions through the nodes is preserved because the sum of in-flows must equal the flow through the inner edge, which must be respect the node capacity by construction. Conservation conditions are preserved because the sum of flows into the in-node must equal the flow through the inner edge, which must equal the sum of flows out of the out-node. The converse is trivial to check. Then, a max flow on the derived graph can be computed using Ford-Fulkerson and translated into a max flow on the derived graph.

A $s-t$ cut in a edge-limited graph can be defined as the partition of $V$ into $3$ sets: $S$, $T$, and $B$, where $s\in S$, $t\in T$, and $B$ may be empty. Additionally, one of $s$ and $t$ may be in $B$, but then either $S$ or $T$ must be empty. The capacity of such a cut is the sum of the node capacities in $B$ plus the sum of all edges from $S$ to $T$. This translates into a $s-t$ cut $A',B'$ in the derived graph by letting $A'=\{n_{in,out}|n\in S\}\cup\{n_{in}|n\in B\}$ and $B'=\{n_{in,out}|n\in T\}\cup\{n_{out}|n\in B\}$. The capacity of the cut $A',B'$ is the same as the capacity of the cut $S,T,B$, so the MFMC theorem on the derived graph along with the equivalence of max flows shown above implies the MFMC theorem for edge-limited graphs.

\end{document}
