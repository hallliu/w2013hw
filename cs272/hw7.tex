\documentclass{article}
\usepackage{geometry}
\usepackage[namelimits,sumlimits]{amsmath}
\usepackage{amssymb,amsfonts}
\usepackage{multicol}
\usepackage{mathrsfs}
\usepackage{graphicx}
\usepackage{alltt}
\usepackage[cm]{fullpage}
\newcommand{\nc}{\newcommand}
\newcommand{\tab}{\hspace*{5em}}
\newcommand{\conj}{\overline}
\newcommand{\dd}{\partial}
\nc{\nn}{\mathbb{N}}
\nc{\pd}[2]{\frac{\partial {#1}}{\partial {#2}}}
\nc{\ep}{\epsilon}
\nc{\co}{\mathscr{C}}
\nc{\nullset}{\varnothing}
\DeclareMathOperator{\opt}{opt}
\begin{document}
Name: Hall Liu

Date: \today 

\subsection*{7.6}
We reduce this to the bipartite matching problem as follows: Let $A$ be the set of switches, and let $B$ be the set of fixtures. For each pair $a\in A$ and $b\in B$, connect them with an edge iff the line from switch $a$ to fixture $b$ does not intersect any walls. This takes $O(n^2m)$ time, as there are $n^2$ pairs to check and each pair takes $O(m)$ time to check. Then, we can solve the bipartite matching problem in $O(n^3)$ time, since there are at most $n^2$ edges in the bipartite graph.

To establish correctness, suppose that we find a bipartite matching. Then, if we wire the matched switches and fixtures, we are guaranteed by our choice of edges that each fixture is visible from its switch. Conversely, if we have an ergonomic wiring, then the bijection given by the wiring will run along the edges on the bipartite graph we constructed, again by our choice of edges.
\subsection*{7.17}
First, we establish two results. Given any $s-t$ path, there must be some edge along the path which has been compromised, as otherwise $t$ would be reachable from $s$. In addition, along any $s-t$ path, if we have a reachable node followed by an unreachable node, the edge between them on the path must be compromised, since otherwise the unreachable node would be reachable via the former one. 

Now, find any acyclic $s-t$ path in $G$. The path defines a total order on the vertices in the path, so we can do something similar to binary search. Start by pinging the middle node (or something close to the middle). If this middle node is reachable, then we know that somewhere further down the path there is a compromised edge, so we consider the path from the middle node to $t$ and look at the midpoint of that. Similarly, if the middle node is unreachable, then we know that there is a compromised edge upstream of the middle node, so we consider the path from $s$ to the middle node and look at the midpoint of that. Proceeding in a manner similar to binary search, we will eventually come upon two adjacent nodes, one reachable and the other unreachable, with an edge going from the reachable node to the unreachable node. We have thus found a compromised edge. Since the path we found has at most $n$ edges, this takes $O(\log n)$ pings. 

Next, we remove the compromised edge from $G$ and find another path from $s$ to $t$ in the modified graph. Since the removal of one edge reduces the min-cut by at most $1$, we are guaranteed to have another path. We can then iterate the above procedure $k$ times, each time removing the compromised edge found, until we have found all $k$ compromised edges. This takes $O(k\log n)$ pings. Now, since the set of compromised edges is a cut, we just need to remove them all from $G$ and find the connected component containing $t$ to find all unreachable nodes.
\subsection*{8.1}
a. Yes, since vertex cover is NP-complete and interval scheduling is P therefore NP.

\noindent b. Unknown, since independent set is NP-complete and it being reducible to interval scheduling is equivalent to P=NP.
\subsection*{8.2}
This problem is NP because we can check all pairs of customers in S to see if they've bought the same thing, which takes polynomial time. We reduce independent set to this problem. Given a graph, let each of its vertices be customers, and let each of its edges be a product. Let customer $i$ buy one of product $j$ iff edge $j$ is incident on vertex $i$. Then, solving the diverse subset problem on this array is equivalent to solving the independent set problem on the original graph.

Suppose the graph has an independent set of size $k$. Then, no edge connects two nodes in this set, which is equivalent to no product is purchased by two customers in the corresponding set of customers, which is equivalent to that it's a diverse subset. 
\subsection*{8.6}
This problem is NP because we can evaluate the clauses on the satisfying assignment to verify that it come out true. We reduce vertex cover to this problem. Given a graph, assign a boolean variable to each of its edges, and let each node represent a clause containing the boolean variables corresponding to the edges incident upon the node. Then, we get a monotone satisfiability problem. If there exists an assignment for which at most $k$ variables are true, we obtain a covering set of edges by including the edges whose corresponding variable is true in the satisfying assignment. The satisfying assignment is equivalent to each clause containing at least one true variable, which is equivalent to each node having at least one edge incident upon it, which constitutes a solution to the vertex cover problem.
\subsection*{8.7}
This problem is NP because we can take the collection of tuples and iterate through them in linear time to look for duplicates. We can reduce 3D matching to this. Given sets $X,Y,Z$ from 3D matching, let $W=X$. For each element $(x_i,y_j,z_k)$ in the collection $\co$, construct an element $(x_i,x_i,y_j,z_k)$ and add it to the collection $\co'$ of 4-tuples. Then, if we can find $n$ of these 4-tuples that don't share any element, chopping off the head will result in a set of $n$ 3-tuples which don't share any elements. Conversely, if we can find such a set of $n$ 3-tuples, the corresponding set of 4-tuples will also not share any elements because any conflicts will originate from the added component, which can't cause any conflict because it's a copy of the first component of the 3-tuple.
\subsection*{8.17}
This problem is NP because if we are given a cycle, we just add up the weights of its edges in linear time and check if it's zero. We reduce subset-sum to this problem. Begin with the complete directed graph on $n+1$ vertices labeled from $0$ to $n$, with each edge impinging upon vertex $i$ having weight $w_i$. Then, remove all edges from vertex $i$ to vertex $j$, where $i>j$. This makes the graph acyclic due to the edge structure enforcing an ordering on any directed path. Finally, add an edge of weight $-W$ from every vertex back to vertex $0$. 

Suppose that we find a zero-weight cycle on this graph. By the previously mentioned acyclic property, this cycle must pass through an edge of weight $-W$. Further, it cannot pass through an edge of weight $w_i$ twice for any $i$, due to the enforced ordering on any path. Thus, all the non-$-W$ edges this cycle passes through form a subset of the $w_i$ summing to $W$. Conversely, suppose that we have a subset $\{w_{i_1},w_{i_2},\ldots,w_{i_k}\}$ summing to $W$, with $i_1<i_2<\cdots<i_k$. Then, consider the cycle given by following $\{0,i_1,i_2,\ldots,i_k\}$. The edge between $0$ and $i_1$ has weight $w_{i_1}$, the edge between $i_1$ and $i_2$ has weight $w_{i_2}$, and so on, and the edge between $i_k$ and $0$ has weight $-W$. Thus, summing these gives $0$, so we have a zero-weight cycle.
\subsection*{13.1}
Assign a color at random to each node. The probability that any edge is not satisfied is $3/9=1/3$, so each edge has probability $2/3$ of being satisfied. The number of satisfied edges therefore follows a binomial distribution with parameters $|E|$ and $2/3$, so its expected value is $(2/3)|E|>(2/3)c*$. 
\subsection*{13.3}
a. Suppose not. Then, two processes in $S$ are in conflict, which means they both got $1$, which is a contradiction because either process would not have entered $S$ because one of the processes it is in conflict with also got $1$. 

If each process has $d$ conflicts, then it enters $S$ with probability $(1/2)^{d+1}$, since the process enters $S$ only in the case of one event out of an event space of size $2^{d+1}$. The expected value of the size of $S$ is thus binomial with parameters $(1/2)^{d+1}$ and $n$, so its expected value is $n(1/2)^{d+1}$.

\noindent b. The probability that any particular process enters $S$ is now $p(1-p)^d$, since the process has probability $p$ of picking $1$ and each of the ones it's in conflict with has probability $1-p$ of picking $0$. Thus, the expected value is $np(1-p)^d$. Differentiating wrt $p$, we have $n(1-p)^d-dnp(1-p)^{d-1}=0\implies n(1-p)=dnp\implies p=\frac{1}{d+1}$, so the maximum expected value is $\frac{nd^d}{(d+1)^{d+1}}$.
\subsection*{13.9}
Reject the first $n/2$ bids, then accept the next bid that is larger than all previous bids seen. If we get to the last buyer, accept his bid. We want to find the probability that the accepted bid is the largest bid. By the law of total probability we have 
\begin{align*}
P(\text{highest bid is accepted})&=\sum_iP(\text{highest bid is accepted and highest bid is bid $i$})\\
&=\sum_iP(\text{bid $i$ is accepted}|\text{bid $i$ is the highest bid})P(\text{bid $i$ is the highest bid})\\
&=\frac{1}{n}\sum_iP(\text{bid $i$ is accepted}|\text{bid $i$ is the highest bid})\\
\end{align*}
For $i$ from $1$ to $n/2$, the summand is zero because we never accept anything from the first half. For $i$ from $n/2+1$ on, given that $b_i$ is the best bid, it is accepted iff no bid between $n/2+1$ and $i-1$, inclusive, exceeds the maximum value of the first $n/2$ bids. This is equivalent to the maximum of the first $i-1$ bids being in the first $n/2$ bids, which has probability $\frac{n/2}{i-1}$. Summing this over $i$ from $n/2+1$ to $n$, we have that the total probability is 
$$\frac{1}{2}\sum_{i=n/2+1}^n\frac{1}{i-1}>\frac{1}{2}\frac{n/2}{n-1}>\frac{1}{4}$$
\subsection*{6.23}

\end{document}
