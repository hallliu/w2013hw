\documentclass{article}
\usepackage{geometry}
\usepackage[namelimits,sumlimits]{amsmath}
\usepackage{amssymb,amsfonts}
\usepackage{multicol}
\usepackage{graphicx}
\usepackage[cm]{fullpage}
\newcommand{\tab}{\hspace*{5em}}
\newcommand{\conj}{\overline}
\newcommand{\dd}{\partial}
\newcommand{\ep}{\epsilon}
\newcommand{\openm}{\begin{pmatrix}}
\newcommand{\closem}{\end{pmatrix}}
\DeclareMathOperator{\cov}{cov}
\newcommand{\nc}{\newcommand}
\newcommand{\rn}{\mathbb{R}}
\begin{document}
Name: Hall Liu

Date: \today 
\vspace{1.5cm}

\subsection*{12.11}
Let all means be zero. This is consistent with all the parameters being zero, so there is no interaction.
\subsection*{12.12}
A three-factor two-level table of cell means can be visualized as a $2\times2\times2$ cube, where each face represents a level of a factor. The main effect of a level of a factor is then the average of the means over a face. If we let the means in level I of factor A be $\openm1&-1\\-1&1\closem$ and the means in level II of factor A be $\openm0&0\\0&0\closem$, we can see that the main effect of both factors of $A$ are $0$. Similarly, the faces corresponding to factors of B and C look like $\openm\pm1&\mp1\\0&0\closem$, so their main effects are zero as well. However, since not all the cells are zero, there must be some contribution from interactions.
\subsection*{12.26}
This can be considered an example of a randomized block design. Running an ANOVA with the assumption of zero interaction, we have

\begin{tabular}{cccccc}
&df&ss&ms&F&p\\
\hline
Treatment&2&1.081&0.540&4.406&0.0277\\
Dog&9&0.517&0.0574&0.468&0.877\\
Error&18&2.21&0.122\\
Total&29&3.806\\
\end{tabular}

This indicates that there is a significant difference between the three treatments at the $5\%$ level. Looking at the means of each of the treatments, cyclopropane had a mean of $0.853$, which is considerably higher than the means of isofluorane and halothane, which were $0.434$ and $0.469$, respectively.
\subsection*{12.34}
a. ANOVA table:

\begin{tabular}{cccccc}
&df&ss&ms&F&p\\
\hline
Treatment&3&91.9&30.6&14.0&$3.27\times10^{-6}$\\
Poison&2&103&51.5&23.6&$2.86\times10^{-7}$\\
Interaction&6&24.7&4.12&1.89&0.110\\
Error&36&78.7&2.19\\
Total&47&298\\
\end{tabular}

b. ANOVA table:

\begin{tabular}{cccccc}
&df&ss&ms&F&p\\
\hline
Treatment&3&0.204&0.0680&28.4&$1.34\times10^{-9}$\\
Poison&2&0.349&0.174&72.8&$2.22\times10^{-13}$\\
Interaction&6&0.0157&0.00261&1.09&0.386\\
Error&36&0.0862&0.00239\\
Total&47&0.0654\\
\end{tabular}

In both cases, the F-values for the treatment and the poison were incredibly high. From looking at a summary of the data, there were qualitatively large differences between the means of different poisons and treatments. However, the sample variance was also quite different. For example, in the original data, treatment $A$ has a sample variance of $1.04$, whereas treatment $B$ has a sample variance of $10.29$. This suggests that the standard ANOVA model may not return the best results. However, when looking at the variances of the reciprocals, the difference lessened, with the highest at $0.013$(A) and the lowest at $0.0066$(B). 

In both cases, the interaction was not significant at the $5\%$ level, which suggests that the none of the treatments are specialized for any one poison.
\subsection*{12.35}
ANOVA table:

$$\begin{array}{c|ccccc}
&df&ss&ms&F&p\\
\hline
\text{Serum type}&2&6.30\times10^8&3.15\times10^8&34.8&4.50\times10^{-10}\\
\text{Dose}&7&2.82\times10^9&4.03\times10^8&44.5&7.28\times10^{-19}\\
\text{Interaction}&14&3.22\times10^8&2.30\times10^7&2.54&0.00833\\
\text{Error}&48&4.34\times10^8&9.04\times10^6\\
\text{Total}&71&4.21\times10^9\\
\end{array}$$

From the ANOVA, it seems that the main effects for both sources and the interaction are significant at the $95\%$ level. In particular, we are now interested in whether the PEG serum is significantly better than the untreated serum. To do so, we conduct a paired $t$-test for $H_0:\mu_{PEG}=\mu_{untreated}$ versus $H_a:\mu_{PEG}>\mu_{untreated}$. We choose the pairs arbitrarily from each dose level -- for simplicity's sake, the pairs will just be the measurements that appear next to each other in the book. Then, we get a $t$-statistic of $\frac{2581.11-0}{4931.66/4.899}=2.56$. Under a $t$-distribution with $23$ degrees of freedom, this corresponds to a one-sided $p$-value of $0.0087$, which is significant at the $95\%$ level. 

We could also run another ANOVA with the control group omitted. The results, shown below, also show a significant effect from the serum type, indicating that our $t$-test from above was accurate.

$$\begin{array}{c|ccccc}
&df&ss&ms&F&p\\
\hline
\text{Serum type}&1&7.99\times10^7&7.99\times10^7&11.0&0.00231\\
\text{Dose}&7&1.42\times10^9&2.04\times10^8&27.9&6.64\times10^{-12}\\
\text{Interaction}&7&1.52\times10^8&2.17\times10^7&2.97&0.0161\\
\text{Error}&32&2.33\times10^8&7.29\times10^6\\
\text{Total}&47&1.89\times10^9\\
\end{array}$$

\subsection*{12.19}
Let $Y_{ijkl}=\mu+\alpha_i+\beta_j+\gamma_k+\delta^{\alpha\beta}_{ij}+\delta^{\alpha\gamma}_{ik}+\delta^{\beta\gamma}_{jk}+\delta^{\alpha\beta\gamma}_{ijk}+\ep_{ijkl}$. $\alpha,\beta$, and $\gamma$ represent the main effects of $A,B$, and $C$ resp. The deltas represent interactions between the factors in the superscript, and the $\ep$ represents the random fluctuations within each cell. The two-factor interactions can be interpreted as some special effect resulting when the two factors are present, but no influence beyond the main effect is derived from the third factor. We have the constraints $\sum_i\alpha_i=0,\sum_j\beta_j=0,\sum_k\gamma_k=0$ on the main factors, as any nonzero sum can be absorbed into $\mu$. On the two-factor interactions, we have $\sum_i\delta^{\alpha\beta}_{ij}=0$ for each $j$ and $\sum_j\delta^{\alpha\beta}_{ij}=0$ for each $i$. This is because any nonzero sum can be absorbed into $\alpha_i$ or $\beta_j$. This general constraint form also holds for the other two two-factor interactions. For the three-factor interactions, we have that the sum over any index should come out to zero because a nonzero sum can be absorbed into the proper main interaction, and the sum over any two indices should come out to zero because a nonzero sum can be absorbed into the proper two-factor constraint. Thus, if we have $I$ levels of $A$, $J$ levels of $B$ and $K$ levels of $C$, we get $3$ main constraints, $2(I+J+K)$ two-factor constraints, and $IJ+JK+IK+I+J+K$ three-factor constraints.
\end{document}
