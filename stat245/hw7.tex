\documentclass{article}
\usepackage{geometry}
\usepackage[namelimits,sumlimits]{amsmath}
\usepackage{amssymb,amsfonts}
\usepackage{multicol}
\usepackage{graphicx}
\usepackage[cm]{fullpage}
\newcommand{\tab}{\hspace*{5em}}
\newcommand{\conj}{\overline}
\newcommand{\dd}{\partial}
\newcommand{\ep}{\epsilon}
\newcommand{\openm}{\begin{pmatrix}}
\newcommand{\closem}{\end{pmatrix}}
\DeclareMathOperator{\cov}{cov}
\newcommand{\nc}{\newcommand}
\newcommand{\rn}{\mathbb{R}}
\begin{document}
Name: Hall Liu

Date: \today 
\vspace{1.5cm}

\subsection*{12.11}
Let all means be zero. This is consistent with all the parameters being zero, so there is no interaction.
\subsection*{12.12}
A three-factor two-level table of cell means can be visualized as a $2\times2\times2$ cube, where each face represents a level of a factor. The main effect of a level of a factor is then the average of the means over a face. If we let the means in level I of factor A be $\openm1&-1\\-1&1\closem$ and the means in level II of factor A be $\openm0&0\\0&0\closem$, we can see that the main effect of both factors of $A$ are $0$. Similarly, the faces corresponding to factors of B and C look like $\openm\pm1&\mp1\\0&0\closem$, so their main effects are zero as well. However, since not all the cells are zero, there must be some contribution from interactions.
\subsection*{12.26}
We use the one-way layout by averaging over each treatment -- the two-way is intractable because $ss_E$ has zero degrees of freedom here. However, the F-test may not be appropriate here -- the sample variance of the concentrations under cyclopropane is $0.2$, which is quite a bit higher than the sample variance of the other two. An ANOVA table follows

\begin{tabular}{cccccc}
&df&ss&ms&F&p\\
\hline
Between&2&1.081&0.540&5.354&0.011\\
Within&27&2.725&0.101\\
Total&29&3.806\\
\end{tabular}

Looking at the means of each of the treatments, cyclopropane had a mean of $0.863$, which is considerably higher than the means of isofluorane and halothane, which were $0.434$ and $0.469$, respectively.
\subsection*{12.34}
a. ANOVA table:

\begin{tabular}{cccccc}
&df&ss&ms&F&p\\
\hline
Treatment&3&91.9&30.6&14.0&$3.27\times10^{-6}$\\
Poison&2&103&51.5&23.6&$2.86\times10^{-7}$\\
Interaction&6&24.7&4.12&1.89&0.110\\
Error&36&78.7&2.19\\
Total&47&298\\
\end{tabular}

b. ANOVA table:

\begin{tabular}{cccccc}
&df&ss&ms&F&p\\
\hline
Treatment&3&0.204&0.0680&28.4&$1.34\times10^{-9}$\\
Poison&2&0.349&0.174&72.8&$2.22\times10^{-13}$\\
Interaction&6&0.0157&0.00261&1.09&0.386\\
Error&36&0.0862&0.00239\\
Total&47&0.0654\\
\end{tabular}

In both cases, the F-values for the treatment and the poison were incredibly high. From looking at a summary of the data, there were qualitatively large differences between the means of different poisons and treatments. However, the sample variance was also quite different. For example, in the original data, treatment $A$ has a sample variance of $1.04$, whereas treatment $B$ has a sample variance of $10.29$. This suggests that the standard ANOVA model may not return the best results. However, when looking at the variances of the reciprocals, the difference lessened, with the highest at $0.013$(A) and the lowest at $0.0066$(B). 

In both cases, the interaction was not significant at the $5\%$ level, which suggests that the none of the treatments are specialized for any one poison.
\subsection*{12.35}
ANOVA table:

$$\begin{array}{c|ccccc}
&df&ss&ms&F&p\\
\hline
\text{Serum type}&2&6.30\times10^8&3.15\times10^8&34.8&4.50\times10^{-10}\\
\text{Dose}&7&2.82\times10^9&4.03\times10^8&44.5&7.28\times10^{-19}\\
\text{Interaction}&14&3.22\times10^8&2.30\times10^7&2.54&0.00833\\
\text{Error}&48&4.34\times10^8&9.04\times10^6\\
\text{Total}&71&4.21\times10^9\\
\end{array}$$

\end{document}
