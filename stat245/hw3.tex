\documentclass{article}
\usepackage{geometry}
\usepackage[namelimits,sumlimits]{amsmath}
\usepackage{amssymb,amsfonts}
\usepackage{multicol}
\usepackage{graphicx}
\usepackage[cm]{fullpage}
\newcommand{\tab}{\hspace*{5em}}
\newcommand{\conj}{\overline}
\newcommand{\dd}{\partial}
\setlength{\parindent}{0mm}
\newcommand{\ep}{\epsilon}
\newcommand{\openm}{\begin{pmatrix}}
\newcommand{\closem}{\end{pmatrix}}
\DeclareMathOperator{\cov}{cov}
\DeclareMathOperator{\var}{var}
\newcommand{\nc}{\newcommand}
\newcommand{\rn}{\mathbb{R}}
\begin{document}
Name: Hall Liu

Date: \today 
\vspace{1.5cm}

\subsection*{1}
a,b.By Example C in section 3.5, the conditional distribution $X|Y$ is normal with mean $\mu_X+\rho(y-\mu_Y)\sigma_X/\sigma_Y$ and variance $\sigma_X^2(1-\rho^2)$. Plugging the given values and $y=8$, we have that the conditional mean is $6.5$ and the conditional variance is $7.56$.

c. We find this probability by integrating the joint density over the strip $[3,5]\times[-\infty,\infty]$, which is really just integrating up the marginal density of $X$ over $[3,5]$. This gives $0.088$.

d. Integrating the conditional distribution over this range gives $0.099$.
\subsection*{2}
a. If we let $A,B$ be iid standard normal distributions, then taking $\openm X\\Y\closem=\openm3&0\\4\rho &4\sqrt{1-\rho^2}\closem\openm A\\B\closem+\openm3\\1\closem$ will yield $X$ and $Y$ with the parameters that we want. This follows directly from the formulas for computing correlation and variances for linear combinations of RVs. Now, we have $\openm W_a\\V\closem=\openm a&1\\1&2\closem\openm X\\Y\closem+\openm12\\9\closem=\openm3a+4\rho&4\sqrt{1-\rho^2}\\3+8\rho&8\sqrt{1-\rho^2}\closem\openm A\\B\closem+\openm3a+13\\14\closem$. The vector at the end there doesn't really matter, as it only tweaks with the mean. Recall that the covariance of two RVs given by a linear combination $\openm a&b\\c&d\closem$ of two standard normal RVs is $ac+bd$, so if we want to make $W_a$ and $V$ independent, we need $(3a+4\rho)(3+8\rho)+32(1-\rho)^2=0$, or $a=-\frac{4(3\rho+8)}{3(8\rho+3)}$. For $\rho=1/3$, this yields $a=-36/11$. If $\rho_0=-3/8$, no value of $a$ will make the two independent.

b. Starting again from $(3a+4\rho)(3+8\rho)+32(1-\rho)^2=0$, we have $\rho=-\frac{32+9a}{24a+12}=-41/36$ for $a=1$. This is outside of the domain of $\rho$, so this $a$ doesn't work. In fact, for all $a$ such that $\frac{32+9a}{24a+12}>1$ or $\frac{32+9a}{24a+12}<-1$, there exists no $\rho$ that will work. This works out to the invalid values being $[-4/3,4/3]$.
\subsection*{3}
a. Let $K$ be a 3-dimensional random variable with independent standard normal components. Then, we have 
$$\openm S\\E_1\\E_2\closem=\openm7&0&0\\0&5&0\\0&0&5\closem K+\openm70\\0\\0\closem$$
Introduce the new variable $Z=-5S/7+7E_1/5+7E_2/5$ (yes, really random, but it makes things look nice later). Then, we have 
$$\openm X\\Y\\Z\closem=\openm1&1&0\\1&0&1\\-5/7&7/5&7/5\closem\openm S\\E_1\\E_2\closem=\openm7&5&0\\7&0&5\\-5&7&7\closem K+\openm70\\70\\-50\closem$$
Let that matrix in the last equation be $A$. Then, by a theorem given in class, we have that the joint density of $X$, $Y$, and $Z$ is 
$$\frac{1}{(2\pi)^{3/2}\sqrt{|\det(AA^T)|}}\exp\left(-\frac{1}{2}\langle x-\mu|A^{-T}A^{-1}|x-\mu\rangle\right)$$
where $x$ is the vector consisting of $X,Y$, and $Z$. We have 
$$AA^T=\openm74&49&0\\49&74&0\\0&0&123\closem\quad\text{and}\quad (AA^T)^{-1}=\openm\frac{74}{3075}&-\frac{49}{3075}&0\\-\frac{49}{3075}&\frac{74}{3075}&0\\0&0&\frac{1}{123}\closem$$
This is nice. These things are block-diagonal, which means that we'll be able to separate the $Z$ term out in the exponential without doing any annoying integration. In fact, the determinant out front will spit out the right term for the $Z$ variable, which means we can just write the joint distribution of $X$ and $Y$ as 
$$\frac{1}{2\pi\sqrt{3075}}\exp\left(-\frac{1}{2}\openm X-70&Y-70\closem\Sigma^{-1}\openm X-70\\Y-70\closem\right)$$
where $\Sigma=\openm74&49\\49&74\closem$

b. The matrix $\Sigma$ above provides us that the variance of $X$ and $Y$ are both $74$, and their correlation is $49/74$. Using the formula given in the book again to determine the conditional distribution of $Y$ given $X=70-\sqrt{74}$, we have that the conditional mean is $64.3$.

c. Two standard deviations above the midterm mean is $x=70+2\sqrt{74}$, so the conditional distribution of $Y$ has mean $81.39$ and variance $41.55$. The probability that such a student will receive a score at least two standard deviations above the final mean (same as the midterm mean) is $0.44$. Thus, more than $50\%$ of the students who were two standard deviations above average on the midterm are expected to score below that on the final. It thus seems plausable that ``many'' of the high-scoring students would score below the two standard deviation mark.
\subsection*{4}
We can express $Y_1$ and $Y_2$ as $\openm Y_1\\Y_2\closem=\openm1&0\\1&1\closem\openm X_1\\X_2\closem$, where $X_1$ and $X_2$ are independent standard normal RVs. Then, to find a linear combination of the $Y$s such that they result in indep. std. normal RVs, all we have to do is invert that matrix up there. It's got determinant $1$, so this gives us $\openm 1&0\\-1&1\closem$.
\subsection*{5}
I couldn't think of an easy way to do this, so I'm doing it the hard way. Using the definition of correlation, we want to get $\frac{\cov(X^2,Y^2)}{\var(X^2)\var(Y^2)}=\frac{E(X^2Y^2)-E(X^2)E(Y^2)}{\var(X^2)\var(Y^2)}$. We know that $E(X^2)=E(Y^2)=\sigma^2_X=\sigma^2_Y=4$, since the means of $X$ and $Y$ are both $0$. We know the marginal densities of $X$ and $Y$, so we can calculate $\var(X^2)=\var(Y^2)=E(X^4)-(E(X^2))^2$. From Wikipedia, $E(X^4)=3!!\sigma_X^4=48$, so $\var(X^2)=32$. Now, we just need to calculate $E(X^2Y^2)$.

We need to compute the integral $\int_{\rn^2}x^2y^2f(x,y) dx dy$. First, let's look at the inside of the exponential and make it nicer. We have 
$$-\frac{1}{2(1-\rho^2)}\left(\frac{x^2}{4}+\frac{y^2}{4}-\frac{\rho xy}{2}\right)$$
Complete the square in $x$ by adding and subtracting $y^2/4-\rho^2y^2/4$ to get 
$$-\frac{1}{2(1-\rho^2)}\left(\frac{x}{2}-\frac{y\rho}{2}\right)^2+\frac{y^2}{8}$$
Now put everything up into the exponential and examine the integral in $x$. We have
$$\frac{y^2e^{-y^2/8}}{8\pi\sqrt{1-\rho^2}}\int_{-\infty}^{\infty}x^2\exp\left(-\frac{1}{2(1-\rho^2)}\left(\frac{x}{2}-\frac{y\rho}{2}\right)^2\right)dx$$
Substituting $u=\frac{1}{\sqrt{2(1-\rho^2)}}(x/2-y\rho/2)$, we incur a factor of $\sqrt{8(1-\rho^2)}$ in front, and the integral itself turns into 
$$\int_{-\infty}^{\infty}(2u\sqrt{2(1-\rho^2)}-y\rho)^2e^{-u^2}du$$
Fortunately, a table of integrals informs me that the integral over the real line of $(au+b)e^{-u^2}$ is equal to $\frac{\sqrt{\pi}}{2}(a^2+2b^2)$, so this integral is actually $\sqrt{\pi}(4(1-\rho^2)+y^2\rho^2)$. Putting this together with the terms in front, we have that the $y$ integral is 
$$\frac{1}{8\sqrt{\pi}}\int_{-\infty}^\infty\sqrt{8}(y^2\rho^2+4(1-\rho^2))y^2e^{-y^2/8} dy$$
Making the substitution $u=y/\sqrt{8}$, we incur a factor of $\sqrt{8}$ out front and we get the integral 
$$\frac{8}{\sqrt{\pi}}\int_{-\infty}^\infty(8u^2\rho^2+4(1-\rho^2))u^2e^{-u^2}du$$
The table of integrals informs me that the integral of $u^2e^{-u^2}$ is $\frac{\sqrt{\pi}}{2}$ and the integral of $u^4e^{-u^2}$ is $\frac{3\sqrt{\pi}}{4}$, so the integral above is now essentially solved in two pieces, with the $u^4$ piece evaluating to $6\rho^2\sqrt{\pi}$ and the $u^2$ piece evaluating to $2(1-\rho^2)\sqrt{\pi}$. Thus, we have that the final result is $32\rho^2+16$. Subtracting off the $E(X^2)E(Y^2)$ term gives $32\rho^2$, and dividing by $32^2$ gives 
$$\frac{\rho^2}{32}$$
\subsection*{6}
The book gives a construction of joint RVs using copulas. Let $c(x,y)=2x+2y-4xy$. This is shown on p77 to have uniform marginals. Let $X,Y$ be standard normal RVs, and denote their density by $\phi$ and cumulative density by $\Phi$. Then, the density function $(2\Phi(x)+2\Phi(y)-4\Phi(x)\Phi(y))\phi(x)\phi(y)$ has normal marginals, but it's not bivariate normal because it can't be expressed in terms of elementary functions.
\subsection*{7}
From Example A on p340, the LRT rejects for large values of $\frac{n}{\sigma^2}(\bar{X}-\mu_0)^2$. Call this $L$, which is distributed according to $\chi^2_1$. From Theorem B in Section 6.3, we have that $K=\frac{(n-1)S^2}{\sigma^2}$ is distributed according to $\chi^2_{n-1}$. Then, substituting for $\sigma_2$, we have that $L=\frac{nK(\bar{X}-\mu_0)^2}{(n-1)S^2}$ or $\frac{\sqrt{n}(\bar{X}-\mu_0)}{S}=\sqrt{\frac{L}{K/(n-1)}}$. Note that the RHS has a $t$-distribution with $n-1$ degrees of freedom by definition. Now, note that $s_{\bar{X}}=\frac{S}{\sqrt{n}}$, and we have that the two tests are equivalent.
\end{document}
