\documentclass{article}
\usepackage{geometry}
\usepackage[namelimits,sumlimits]{amsmath}
\usepackage{amssymb,amsfonts}
\usepackage{multicol}
\usepackage{graphicx}
\newcommand{\tab}{\hspace*{5em}}
\newcommand{\conj}{\overline}
\newcommand{\dd}{\partial}
\setlength{\parindent}{0mm}
\newcommand{\ep}{\epsilon}
\begin{document}
Name: Hall Liu

Date: \today 
\vspace{1.5cm}

\subsection*{8.1}
The number of intervals seems to have gone up in the data we're given in this problem. 

The value given for the emission rate was $\hat{\lambda}=0.8392\text{s}^{-1}$, and we had $12169$ intervals. Using these numbers, we get these expected values:

\begin{tabular}{c|c}
n & Expected\\
\hline
0 & 5257\\
1 & 4412\\
2 & 1851\\
3 & 518\\
4 & 109\\
5+ & 22\\
\end{tabular}

Looks pretty good. About $2\%$ error on most of them.

\subsection*{8.2}
Total number of intervals is $300$. Total count is found by multiplying frequency by count and summing, calculated to be $1168$. We estimate the rate parameter $\hat{\lambda}=3.893$. Expected frequencies can be found by multiplying the Poisson probability by the number of intervals. Results below.

\begin{tabular}{c|c|c}
n & Observed & Expected\\
\hline
0 & 14 & 6\\
1 & 30 & 24\\
2 & 36 & 46\\
3 & 68 & 60\\
4 & 43 & 59\\
5 & 43 & 46\\
6 & 30 & 30\\
7 & 14 & 16\\
8 & 10 & 8\\
9 & 6 & 3\\
10 & 4 & 1\\
11 & 1 & 0\\
12 & 1 & 0\\
\end{tabular}

The predicted frequencies seems to have a lower variance than the observed data. This is probably due to varying levels of traffic over the times that the data were taken.

\subsection*{8.3}
a. Concentration 1: 0.6825; Concentration 2: 1.323; Concentration 3: 1.80; Concentration 4: 4.68

b. From use of the variance-stabilizing transformation in class, we have $2\sqrt{n}(\sqrt{\bar{X}}-\sqrt{\lambda})~N(0,1)$, or that $P(-1.96\leq 2\sqrt{n}(\sqrt{\bar{X}}-\sqrt{\lambda})\leq1.96)$. Rearranging for $\lambda$, we get a confidence interval of $[(-0.98/\sqrt{n}+\sqrt{\bar{X}})^2,(0.98/\sqrt{n}+\sqrt{\bar{X}})^2]$. Plugging in the values we actually have, this is 

Concentration 1: $[0.604,0.766]$; Concentration 2: $[1.212,1.438]$; Concentration 3: $[1.671,1.934]$; Concentration 4: $[4.47,4.894]$

c. Format in each cell of the table is observed,expected. Things generally look pretty good.

\small
\begin{tabular}{c|c|c|c|c}
\hline
\#Cells & C1 & C2 & C3 & C4\\
\hline
0 & 213,202 & 103,107 & 75,66 & 0,4 \\
1 & 128,138 & 143,141 & 103,119 & 20,17 \\
2 & 37,47 & 98,93 & 121,107 & 43,41 \\
3 & 18,11 & 42,41 & 54,64 & 53,63 \\
4 & 3,2 & 8,14 & 30,29 & 86,74 \\
5 & 1,0 & 4,4 & 13,10 & 70,69 \\
6 & 0,0 & 2,1 & 2,3 & 54,54 \\
7 & 0,0 & 0,0 & 1,1 & 37,36 \\
8 & 0,0 & 0,0 & 0,0 & 18,21 \\
9 & 0,0 & 0,0 & 1,0 & 10,11 \\
10 & 0,0 & 0,0 & 0,0 & 5,5 \\
11 & 0,0 & 0,0 & 0,0 & 2,2 \\
12 & 0,0 & 0,0 & 0,0 & 2,1 \\
\hline
\end{tabular}
\normalsize
\subsection*{8.10}
Using the normal approximation, we have $P(\hat{\lambda}=v)=P(S=nv)=f_{n\lambda_0,n\lambda_0}(nv)=f_{\lambda_0,\lambda_0/n}(v)$, where $f_{\lambda_0,\lambda_0/n}$ is the normal pdf with mean $\lambda_0$ and variance $\lambda_0/n$. The probability calculation we're asked to do is then a calculation for a normal RV. In order to approximate the true mean, we use the sample mean $\hat{\lambda}$.

\begin{tabular}{c|c}
$\delta$ & $P(|\lambda_0-\lambda|>\delta)$\\
\hline
0.5 & 0.644\\
1 & 0.356\\
1.5 & 0.166\\
2 & 0.0647\\
2.5 & 0.0209\\
\end{tabular}

\includegraphics[width=0.7\textwidth]{scripts_hw1/8_10.png}

\subsection*{5}
Let $W$ be a Bernoulli RV taking the value $1$ with probability $e^{-\lambda}$ (corresponding to a zero) and $0$ otherwise (corresponding to nonzero). The expected value of $W$ is $e^{-\lambda}$, so consider $W'=W-e^{-\lambda}$ so we can use CLT. $W$ has variance $\rho^2=e^{-\lambda}(1-e^{-\lambda})$, so so does $W'$. By CLT, the asymptotic distribution of $\frac{\sum{W'_i}}{\rho\sqrt{n}}=\frac{Y}{\rho\sqrt{n}}-\frac{ne^{-\lambda}}{\rho\sqrt{n}}$ is the standard normal distribution. Thus, $Y$ has normal asymptotic distribution with mean $ne^{-\lambda}$ and variance $n\rho^2$, so $Y/n$ is asymptotically normal with mean $\mu=e^{-\lambda}$ and variance $\sigma^2=\rho^2/n$.

Now take the series expansion for log about $\mu$. Since the variance in $Y/n$ decreases with increasing $n$, the higher-order terms in $Y/n-\mu$ will become negligible fairly quickly, so we take the first two terms. This is $-\log\mu-\frac{Y/n-\mu}{\mu}=1+\lambda-\frac{Y/n}{\mu}$. This is still normal with mean $\mu'=\lambda$ and variance $\sigma'^2=1/n(e^{-\lambda}-1)$

The variance of this estimator decreases with $n$ rather than increasing with $n$ like the MLE. 
\end{document}
