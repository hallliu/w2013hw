\documentclass{article}
\usepackage{geometry}
\usepackage[namelimits,sumlimits]{amsmath}
\usepackage{amssymb,amsfonts}
\usepackage{multicol}
\usepackage{graphicx}
\usepackage[cm]{fullpage}
\newcommand{\tab}{\hspace*{5em}}
\newcommand{\conj}{\overline}
\newcommand{\dd}{\partial}
\newcommand{\ep}{\epsilon}
\newcommand{\openm}{\begin{pmatrix}}
\newcommand{\closem}{\end{pmatrix}}
\DeclareMathOperator{\cov}{cov}
\newcommand{\nc}{\newcommand}
\newcommand{\rn}{\mathbb{R}}
\begin{document}
Name: Hall Liu

Date: \today 
\vspace{1.5cm}

\subsection*{1}
Let $Z=X$, $W=X/Y$. Then $X=Z$ and $Y=Z/W$, so the Jacobian determinant is $Z/W^2$. The joint distribution of $W,Z$ is 
$$\frac{z}{w^2}e^{-z(1+\frac{1}{w})}$$
We want to integrate $z$ out from this. Noting that this takes the form of a gamma distribution in $z$ with $\lambda=(1+\frac{1}{w})$ and $r=2$, we have that the integral over $[0,\infty)$ in $z$ is $1/(1+\frac{1}{w})^2$. Thus, the distribution of $W$ is
$$\frac{1}{(w+1)^2}$$
If we look at the pdf of the $F_{2,2}$ distribution, then we have 
$$f(w)=\frac{\Gamma(4/2)}{\Gamma(1)\Gamma(1)}\left(\frac{2}{2}\right)^{2/2}w^{2/2-1}\left(1+\frac{2}{2}w\right)^{-(2+2)/2}=\frac{1}{(1+w)^2}$$
so the two are the same distribution.
\subsection*{2}
From Theorem B, we have that $\frac{(n-1)S_x^2}{(m-1)S_y^2}$ is distributed like the quotient of a $\chi_{n-1}^2$ RV by a $\chi_{m-1}^2$ RV. Then, if we divide both sides by $(n-1)$ and multiply by $(m-1)$, we have that $S_x^2/S_y^2$ is distributed according to $F_{n-1,m-1}$. Then, if we know the cdf of the $F$ distribution, we can compute $P(S_x^2/S_y^2>c)$.
\subsection*{3}
a. If we have $H_a:\sigma_X>\sigma_Y$, then $s_X^2/s_Y^2$ would be distributed according to $\frac{\sigma_X^2}{\sigma_Y^2}W$, where $W\sim F_{n-1,m-1}$ by a similar derivation as in 2. Thus, we want to reject for high values of $s_X^2/s_Y^2$, values for which $1-\text{cdf}(F_{n-1,m-1})(s_X^2/s_Y^2)<\alpha$. Similarly, rejection region for the other one-sided alternative is $\text{cdf}(F_{n-1,m-1})(s_X^2/s_Y^2)<\alpha$, and for the two-sided alternative it is the union of the two above regions, but with $\alpha$ replaced by $\alpha/2$. 

\noindent b. Let $w(p)$ be the inverse cdf of the $F_{n-1,m-1}$ distribution, and let $r=\sigma_X^2/\sigma_Y^2$. We have that $r^{-1}\frac{s_X^2}{s_Y^2}\sim F_{n-1,m-1}$, so we'll reject when $r^{-1}\frac{s_X^2}{s_Y^2}>w(1-\alpha/2)$ or when $r^{-1}\frac{s_X^2}{s_Y^2}<w(\alpha/2)$, so the confidence interval is $$\left[\frac{s_X^2}{w(1-\alpha/2)s_Y^2},\frac{s_X^2}{w(\alpha/2)s_Y^2}\right]$$

\noindent c. We want to test if $\sigma_A=\sigma_B$. The statistic in question is $s_A^2/s_B^2=0.599$. We have $13$ samples in $A$ and $8$ in $B$, so this statistic follows $F_{12,7}$. The cdf at this point is $0.208$, so we do not reject. To construct the confidence interval, we calculate the inverse cdf at $p=0.025$ and $p=0.975$, which come out to be $0.277$ and $4.666$, resp. Then, the confidence interval for $s_A^2/s_B^2$ is $[0.128,2.162]$
\subsection*{4}
$\bar{D}=\bar{X}-\bar{Y}=0.44$, and $s_{\bar{D}}=4.63$. If we ignored the pairing, we'd have that $s_{\bar{X}-\bar{Y}}=s_p\sqrt{1/n+1/m}=7.805$. Since the differences are independent and distributed normally, we have from results given in the book that a $95\%$ percent confidence interval is given by $\bar{D}\pm t_{14}(0.025)s_{\bar{D}}=[-9.492,10.372]$. Thus, there is no significant difference in the mean of the two samples. 
\subsection*{5}
First, compute the sample correlation using 
$$\hat{\rho}=\frac{\frac{1}{n}\sum(X_i-\bar{X})(Y_i-\bar{Y})}{\sqrt{\frac{1}{n}\sum(X_i-\bar{X})^2}\sqrt{\frac{1}{n}\sum(Y_i-\bar{Y})^2}}$$
Running the data through, we get $0.9767$. From class notes, if we use the variance-stabilizing transformation $f(\rho)=1/2\log\left(\frac{1+\rho}{1-\rho}\right)$, we get that $\sqrt{n}(f(\hat{\rho})-f(\rho))\sim N(0,1)$ for large enough $n$. We can then construct a $95\%$ CI for $f(\rho)$ as $f(\hat{\rho})\pm1.96/\sqrt{15}=[1.7137,2.7258]$. Inverting $f$, we have that a $95\%$ CI for $\rho$ is $[0.9371,0.9915]$
\subsection*{6}
We have $s_p^2=\frac{(n-1)s_X^2+(n-1)s_Y^2}{2(n-1)}=\frac{s_X^2+s_Y^2}{2}$ when $m=n$. Multiplying this by $\frac{2}{n}$ gives $\frac{s_X^2+s_Y^2}{n}=\frac{s_X^2}{n}+\frac{s_Y^2}{n}$.
\subsection*{7}
We want to test if $\mu_X=\mu_Y$. The two sets of data should be independent: there is no notion of pairing in the experimental design. Let $X$ denote the controls and let $Y$ denote the ozone-exposed group. Then, we have $\bar{X}=22.4261,\bar{Y}=11.0091,s_X^2=116.138,s_Y^2=361.65$. The sample variances are pretty different, so we can't use the pooled variance method.  Instead, we use the statistic $\frac{\bar{X}-\bar{Y}}{\sqrt{s_X^2/n_Y+s_Y^2/n_Y}}=2.4629$. The dof is given by that long formula which I shall not repeat here, but it comes out to be $6235$. Then, the $p$-value for the statistic we calculated is $0.007$, which is presumably significant. 
\subsection*{8}
The estimator used in class is 
$$\hat{\rho}=\frac{\frac{1}{n}\sum(X_i-\bar{X})(Y_i-\bar{Y})}{\sqrt{\frac{1}{n}\sum(X_i-\bar{X})^2}\sqrt{\frac{1}{n}\sum(Y_i-\bar{Y})^2}}$$
Since we're assuming that the means are zero, if we take large enough samples we can just ignore the $\bar{X}$ and $\bar{Y}$ terms, so we can instead use 
$$\hat{\rho}=\frac{\frac{1}{n}\sum X_iY_i}{\sqrt{\frac{1}{n}\sum X_i^2}\sqrt{\frac{1}{n}\sum Y_i^2}}$$
This is not a MLE. To see this, we first find the likelihood function for the bivariate data. We have 
$$L(\rho)=\left(\frac{1}{2\pi\sqrt{1-\rho^2}}\right)^n\exp\left(-\frac{1}{2(1-\rho^2)}\sum_{i=1}^n(X_i^2+Y_i^2-2\rho X_iY_i)\right)$$
and the log-likelihood
$$l(\rho)=-n\log(2\pi\sqrt{1-\rho^2})-\frac{1}{2(1-\rho^2)}\sum_{i=1}^n(X_i^2+Y_i^2-2\rho X_iY_i)$$
Denote $\sum X_i^2$ by $a_{xx}$, $\sum Y_i^2$ by $a_{yy}$, and $\sum X_iY_i$ by $a_{xy}$. Then, we have
$$l(\rho)=-n(\log(2\pi)+1/2\log(1-\rho^2))-\frac{1}{2(1-\rho^2)}(a_{xx}+a_{yy}-2\rho a_{xy})$$
Taking the derivative wrt $\rho$ and setting equal to $0$, we get
$$\frac{n\rho}{1-\rho^2}+\frac{a_{xy}}{1-\rho^2}-\frac{\rho}{(1-\rho^2)^2}(a_{xx}+a_{yy}-2\rho a_{xy})=0$$
$$n\rho(1-\rho^2)+a_{xy}(1-\rho^2)-\rho(a_{xx}+a_{yy}-2\rho a_{xy})=0$$
$$-n\rho^3+a_{xy}\rho^2+(n-a_{xx}-a_{yy})\rho+a_{xy}=0$$
Evidently, $\hat{\rho}$ from earlier is not a root of this polynomial, since it has no $n$ dependence.
\end{document}
