\documentclass{article}
\usepackage{geometry}
\usepackage[namelimits,sumlimits]{amsmath}
\usepackage{amssymb,amsfonts}
\usepackage{multicol}
\usepackage{graphicx}
\usepackage{caption}
\usepackage{subcaption}
\usepackage{placeins}
\usepackage[cm]{fullpage}
\newcommand{\tab}{\hspace*{5em}}
\newcommand{\conj}{\overline}
\newcommand{\dd}{\partial}
\newcommand{\ep}{\epsilon}
\newcommand{\openm}{\begin{pmatrix}}
\newcommand{\closem}{\end{pmatrix}}
\DeclareMathOperator{\cov}{cov}
\newcommand{\nc}{\newcommand}
\newcommand{\rn}{\mathbb{R}}
\begin{document}
Name: Hall Liu

Date: \today 
\vspace{1.5cm}

\subsection*{10.5}
We have $\cov(nF_n(u),nF_n(v))=\cov(\sum_iI_u(X_i),\sum_iI_v(X_i))=\sum_i\sum_j\cov(I_u(X_i),I_v(X_j))$. Note that the term inside is zero unless $i=j$ since $X_i$ and $X_j$ are independent otherwise. Then, we get the sum $\sum_i\cov(I_u(X_i),I_v(X_i))$. We have $\cov(I_u(X_i),I_v(X_i))=E(I_u(X_i)I_v(X_i))-E(I_u(X_i))E(I_v(X_i))$. The second term is given in the book to be $F(u)F(v)$. If we assume WLOG that $v\leq u$, then the first term is $1$ iff $I_v(X_i)=1$, so its expected value is just $F(v)=F(m)$. Thus, we have that $\cov(nF_n(u),nF_n(v))=nF(m)(1-F(u))$ or $\cov(F_n(u),F_n(v))=\frac{1}{n}F(m)(1-F(u))$
\subsection*{10.6.a}
\includegraphics[width=0.6\textwidth]{scripts_5/10_6.png}

The data are on the $x$ axis and the normal quantiles are on the $y$. The line $y=x$ has been drawn. This looks a bit off-normal at the ends, and the direction suggests that it has more spread than a normal distribution.
\subsection*{11.11}
Under $H_0$, we have $\mu_X=\mu+\Delta$ and $\mu_Y=\mu$, where $X$ and $Y$ are independent normal RVs both variance $\sigma^2$. Thus, the $t$-statistic 
$$\frac{(\bar{X}-\bar{Y})-\Delta}{s_p\sqrt{1/n+1/m}}$$
has a $t$-distribution with $n+m-2$ degrees of freedom. Then, we can just determine an appropriate cutoff for the level we're testing at with for this statistic.
\subsection*{11.34}
Let $\delta=\mu_X-\mu_Y$. For both designs, we wish to plot $P(\text{reject }H_0|\mu_X-\mu_Y=\delta)$ versus $\delta$. Set a significance level of $0.05$ for both.

\noindent a. Under $H_a$, $\bar{D}=\bar{X}-\bar{Y}$ is normally distributed with mean $\delta$ and variance $\sigma_D^2=\sigma_X^2+\sigma_Y^2-2\cov(X,Y)=100$. We reject the null hypothesis when $\bar{D}/\sigma_D$ goes outside the interval $(-1.96,1.96)$, so the probability of rejection is $P(\bar{D}>19.6)+P(\bar{D}>-19.6)$.

\includegraphics[width=0.4\textwidth]{scripts_5/11_34a.png}

\noindent b. Same as above, except $\sigma_D^2=\sigma_X^2+\sigma_Y^2=200$ now, so the probability of rejection is now $P(\bar{D}>1.96\sqrt{2})+P(\bar{D}>-1.96)$

\includegraphics[width=0.4\textwidth]{scripts_5/11_34b.png}

The unpaired design is less powerful than the paired design for all values of $\delta$ except $0$.
\subsection*{11.44}
\FloatBarrier
a. Boxplots for the three cases:

\begin{figure}[h]
    \begin{subfigure}[b]{0.33\textwidth}
        \centering
        \includegraphics[width=\textwidth]{scripts_5/11_44/a0h.png}
        \caption{0-hour}
    \end{subfigure}
    \begin{subfigure}[b]{0.33\textwidth}
        \centering
        \includegraphics[width=\textwidth]{scripts_5/11_44/a2h.png}
        \caption{2-hour}
    \end{subfigure}
    \begin{subfigure}[b]{0.33\textwidth}
        \centering
        \includegraphics[width=\textwidth]{scripts_5/11_44/adiff.png}
        \caption{Difference}
    \end{subfigure}
\end{figure}
\FloatBarrier
\noindent b. The standard deviations at 0h are pretty different, so we can't use the pooled $t$-test. At 0h, it also looks like the data isn't very normal, so that could be a problem. At 0h, if we test the equality of the means, the $t$-statistic is $2.222$ with approximate dof $22$, so we get a $p$-value of $0.037$.
At 2h, the $t$-statistic is $2.693$ with approximate dof $24$, so the $p$-value is $0.0127$. The differences have a $t$-statistic of $1.818$ and a dof of $29$, so the $p$-value is $0.794$. If we're at $\alpha=0.05$, there's not enough evidence to suggest that the difference is different between the two groups.

\noindent d. Boxplots for the two cases:
\FloatBarrier
\begin{figure}[h]
    \begin{subfigure}[b]{0.47\textwidth}
        \centering
        \includegraphics[width=\textwidth]{scripts_5/11_44/ctot.png}
        \caption{Total}
    \end{subfigure}
    \begin{subfigure}[b]{0.47\textwidth}
        \centering
        \includegraphics[width=\textwidth]{scripts_5/11_44/cmg.png}
        \caption{mg/kg}
    \end{subfigure}
\end{figure}
\FloatBarrier
These don't look particularly normal -- there are some really big outliers on the high end. 

\noindent e. Normality doesn't seem like a reasonable assumption, but that's okay because we have fairly large sample sizes. The sample variances are quite different, so we still can't use the pooled $t$-test. For the total excretion, the $t$-statistic is $1.940$ and the dof is $20$, so the $p$-value is $0.666$. For the mg/kg ratio, the $t$-statistic is $2.025$ and the dof is $22.5$, so the $p$-value is $0.054$. In both cases, at the $0.05$ level, there is insufficint evvidence to conclude that the two groups are different.
\subsection*{11.37}
a. We should use a paired test here, since the experiment was done in pairs. For ward A, the difference in the improvements of the two groups has mean $0.4944$ and standard deviation $0.8202$ for a $t$-statistic of $0.6028$ with dof $9$, so the $p$-value (one-sided, since we're testing $H_0:\mu_D>0$) is $0.28$. For ward B, $\bar{D}=0.246$ and $s_D=0.479$, so $t=0.5128$ and $p=0.313$. In both cases, the effect of stelazine is not significant at the $0.05$ level. 

b. There's no pairing here, and we can't pool because the variances are pretty different for the two wards under stelazine. The $t$-statistic of ward A versus ward B under stelazine is $2.729$ with dof $12$, for a $p$-value of $0.018$ (two-sided). This indicates a significant difference in the improvement between the wards, with ward A performing better.
\subsection*{10.10}
a. For a uniform distribution, the density becomes $n\binom{n-1}{k-1}x^{k-1}(1-x)^{n-k}$. To find the mean, we multiply this by $x$ and integrate. The resulting expression is
$$n\binom{n-1}{k-1}\int_0^1x^k(1-x)^{n-k}=\frac{n\binom{n-1}{k-1}}{(n+1)\binom{n}{k}}\int_0^1(n+1)\binom{n}{k}x^k(1-x)^{n-k}=\frac{n\binom{n-1}{k-1}}{(n+1)\binom{n}{k}}=\frac{k}{n+1}$$

To calculate variance, we have that the density of $Y_{(k)}=X_{(k)}^2$ is $n/2\binom{n-1}{k-1}y^{k/2-1}(1-y)^{(n-k)/2}$. Doing the same thing as above, we get
$$n/2\binom{n-1}{k-1}\int_0^1y^{k/2}(1-y)^{(n-k)/2}=\frac{n/2\binom{n-1}{k-1}}{(n/2+1)\binom{n+1}{k+1}}\int_0^1(n/2+1)\binom{n+1}{k+1}y^{k/2}(1-y)^{(n-k)/2}=\frac{n\binom{n-1}{k-1}}{(n+2)\binom{n+1}{k+1}}=\frac{k(k+1)}{(n+1)(n+2)}$$
This, minus the mean squared, is 
$$\frac{k(k+1)}{(n+1)(n+2)}-\frac{k^2}{(n+1)^2}=$$
\end{document}
